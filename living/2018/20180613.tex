\section{Wed, Jun 13, 2018}

Here we are in the middle of the week. I often wonder if it enjoys being the
middle of the week. But well it's a weekday and doesn't know any better... so
who's to say? Who really cares!

It is but another day. Another day to let things live and other things go.
There's not reason to let anything get in your way of happiness. There's no
reason to allow it to destroy you. Whatever \st{happy} happiness you can make 
for yourself, go out and do it. If it makes you happy, enjoy it! Is it wrong to
say that? I do not think it is. We are meant to have joy in this life right? I
mean, that's what it says in the book?\footnote{2 Nephi 2:25; Adam fell that 
men might be; and men are, that they might have joy.} Then if I am to have joy
in this life, why can't I actually find and experience joy?

The interesting part about this life, religious leaders would have you think you
are experiencing joy. But in reality you are being opressed with rules and other
requirements to live by. They say that being evil was never joyful, who's to say
exactly what is evil? Words and traditions passed down from our fathers as it
were, are these the rules to live by?

It can be confusing at times. Have you ever been confused about something?
You're told to believe in something, have that blind faith as it were. Blind
obedience doesn't really ever solve anything. To have blind obedience, you do
not know what it is you're obeying or \textit{why} you're obeying it.

If God is not the author of confusion as we're taught,\footnote{1 Corinthians 
14:33; For God is not the author of confusion, but of peace, as in all 
churches of the saints.} Then why is there so much confusion within the religion
one belongs in? Even the \textit{one true church} are people conflicted with
teachings and examples. It doesn't quite add up at times. I don't think it will
ever fully add up, it is what it is...and that is a bit intimidating.

But we are told to ever be marching forward with a steadfastness in Jesus
Christ ever tossing aside evil and living to the standards which the Lord as
provided.\footnote{2 Nephi 31:20; Wherefore, ye must press forward with a 
steadfastness in Christ, having a perfect brightness of hope, and a love of God 
and of all men. Wherefore, if ye shall press forward, feasting upon the word of 
Christ, and endure to the end, behold, thus saith the Father: Ye shall have 
eternal life.} We are told to be perfect. Now, there have been those who clarify
the words of the Lord, they say that we cannot be perfect at this current
present moment in time, but we can try our hardest to do that.

Isn't it a bit unreasonable to be expected to live up to Godlike status while we
are in this mortal coil? Perhaps unreasonable isn't the right term for it.
Unlikely? Expecting too much of people? Something along those lines. To expect a
person to be perfect, or they will surely die...it feels wrong to me. Now of
course we have repentance and ways to get back to God. But if we don't repent of
our sins, which should be a daily process, then we cannot dwell with God because
God doesn't allow unclean things to dwell with Him in Heaven. They are able to
go to other kingdoms, but God will not dwell with them there.\footnote{D\&C 76}

They are all kingdoms of God, just differing in degrees of glory. If you don't
live by the rules 100\% you are cut off from God. There are no exceptions to it.
Leaders will try and tell you there is a grey space in between the black and
white, but to be honest? It doesn't work that way. The scriptures are pretty cut
and dry when it comes to who will be allowed entrance into God's kingdom and who
will be cast into outer darkness.\footnote{ibid.}
\footnote{It makes me wonder at times, if there really is any way to get through 
this life without damage to our own sanity. I mean if you think about it. So 
many restrictions to do this and that. If we do ever get to the highest degree 
of glory, and those restrictions will still be there...will we really want to 
be there? I mean, if we didn't fully understand and grasp them here, what makes 
you think we would grasp them in a different life? The indoctrination runs
deep as it were. The \st{inability} not wanting to upset parents and family
members. Because what they say is truth, and if you don't believe it, they will
drag you kicking and screaming into heaven. Well, what if I don't want to be in
the same ``heaven" as they are? If we don't really see eye to eye in this life,
what makes you believe we'll see eye to eye in the \st{second} third life. The
life to come after this life. Maybe that's where the demands of justice will be
met with Christ's mercy? I do not know.}

For a time, we were taught that we would become Gods. We would achive that which
our Father in Heaven has become. The church has backed away from such teachings, 
and we are now taught that we can live with God someday. These thoughts are 
discussed in the articles section of this document.

It comes back to the whole comment from people I love.

\begin{displayquote}
I will drag you kicking and screaming into heaven if I have to!\footnote{
You don't want to know how many times I've heard that line from my parents. Well
my mother mostly. But yeah, it's a thing. It's a very annoying thing. For if I
were to simply follow what they did out of fear? That wouldn't be right. Fear is
not of God. Putting fear into poeple is not expected from God. That's not how
God works. Yes, God destroyed people in the past because of wickedness. But He
also says we have the agency to choose. So, which is it? Do we actually have
agency to choose and be punished according to our choosing, in which we don't
really have agency at all? Or are we simply to be destroyed. It's another
confusing part of the makeup of it all. Not everything makes sense, I get that.
I understand that. For we don't have all of the knowledge at this moment in time
to do anything about it all. So here we sit, waiting for something to happen.
Waiting for either death and destruction or eternal glory. No matter what we do,
we won't ever be good enough. The grace of Christ makes up for all of that. Yet
again we are told that grace alone will not save us. We have to put forth our
works. It goes back and forth on this topic a few times and just doesn't add up.
I suppose that's another thing which doesn't make sense? So many things just
don't make sense these days. I don't understand it.
}
\end{displayquote}

That's not allowing me to use my agency, which was given by God. I don't
understand how such teachings can be okay for people. To tell me I have no
choice in the matter? To tell me I will be forced into heaven? I don't know
about you, but my immediate gut reaction is to kick back. To say no...that's not
how agency works, that's not how I was raised it works. Well actually that is
how I was raised it to work. Either you do it our way or you are in trouble. We
will disown you and shun you long before you even realize what's going on.

Try being shuned from a church you grew up in. It's not a favorable thing to
happen. My parents didn't take what I did very well. We argued a lot over it. I
was obviously in the ``wrong" because I didn't follow the rules the church had
set forth. Being told your parents will ``kill" you if you ever did anything to
endanger your membership, is not a good fear tactic. It only brings about
fighting.

It's almost as if, the church and parents force you to believe in what they
believe. They claim they are teaching you, and you grow and grasp your own
testimony of it all. But in reality, aren't you just following that which you
saw your parents do? What you see other people do? Perhaps I was never fully
converted to the gosepl through the LDS Church. That doesn't feel like a falling
to me. It is what it is.