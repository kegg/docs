\section{Wed, Jan 23, 2019}

Lawrence E. Corbridge stated the following:

\begin{displayquote}
\begin{enumerate}
\item Is there a God who is our Father?
\item Is Jesus Christ the Son of God, the Savior of the world?
\item Was Joseph Smith a prophet?
\item Is The Church of Jesus Christ of Latter-day Saints the kingdom of God on the 
earth?
\end{enumerate}

If you answer the primary questions, the secondary questions get answered too or they 
pale in significance and you can deal with things you understand and things you don’t 
understand, things you agree with and things you don’t agree with without jumping 
ship altogether.\footnote{What to do with your questions, according to 1 General 
Authority who's an expert on anti-Church materials, The Church News}
\end{displayquote}

Yet just a few years ago, we were told by M. Russel Ballard:

\begin{displayquote}

Gone are the days when a student asked an honest question and a teacher responded, 
``Don’t worry about it!" Gone are the days when a student raised a sincere concern 
and a teacher bore his or her testimony as a response intended to avoid the 
issue.\footnote{The Opportunities and Responsibilities of CES Teachers in the 21st 
Century, LDS.org}
\end{displayquote}

So we are told by an Apostle of the Lord that sincere questions are okay to have.
I've been told this by people in the ward as well. Yet another GA says that if you
just answer the primary questions, the rest of the questions don't matter. Which
seems to be conflicting statements in my opinion.