\section{Second Anointing and Calling and Election Made Sure}

There are some doctrines of the church that are not spoken of. Some doctrines 
have been abandoned by the church, disavowed as prophets speaking as men. 
Other doctrines of the church are simply not spoken about.

On lds.org, there is one doctrine that is mentioned with a strict warning not to 
discuss or answer questions about it. That is the doctrine of the second 
anointing.

In Chapter 19 of Doctrines of the Gospel Teacher Manual, it has the following 
caution:

\begin{displayquote}
Caution: Exercise caution while discussing the doctrine of having our calling 
and election made sure. Avoid speculation. use only the sources given here and 
in the student manual. Do \textit{not} attempt in any way to discuss or answer 
questions about the second anointing.
\end{displayquote}

So, uh what’s that about? That’s the only place on the lds.org website where 
that specific doctrine is mentioned. Now we obviously know what it is, but not 
how it is performed. Knowing what and knowing how are two very different things. 
Now you can google all you wish to know about it, I’m sure it’s been talked 
about. There’s no way it’s been kept secret for as long as it’s been known to 
the people of the earth. It’s out there. You just have to look for it.

 Now, some problems with this. Why would the manual provide a caution to not 
 discuss the second anointing? The second anointing is to seal up someone into 
 heaven. They are basically promised eternal life no matter what. The only thing 
 which could prevent them from going to heaven is murder or denying the Holy 
 Ghost.
 
 The second anointing is given to elite leaders of the church. The ordinance is 
 performed by the washing the feet of the couple in the temple. The couple are 
 anointed on the head with oil, their calling and election is made sure. That 
 is, they are guaranteed a place in the highest level of the Celestial Kingdom. 
 After which, the couple goes to a different room in the temple where the wife 
 washes the husbands feet and then lays her hands on his head and gives him a 
 priesthood blessing as the spirit directs. This is in preparation of death and 
 resurrection.
 
 Okay so that brought about an interesting twist. This is having received the 
 more sure word of prophecy. That is, you’re going past judgement day. Your fate 
 has already been made whole. You are basically judged taking God out of the 
 equation.
 
 First off, only God can judge us. That’s what a final judgement is for isn’t 
 it? How can a prophet of God or a man as it were proclaim that you are free of 
 the blood and sin of such a generation? That you are made clean and a King and 
 Priest unto God? Being a equal to that of Jesus Christ Himself.
 
 Second, we’re told that women cannot hold the priesthood in the church. She 
 cannot perform blessings, ordinances etc. Here during the second anointing this 
 is not he case. Not only does she perform an ordinance but she also performs a 
 blessing, not just a blessing though. It’s a blessing as the spirit directs. 
 So it is approved of God.
 
 I suppose this second part shouldn’t be much of a shock to me. Women have been 
 performing ordinances in the temple for years to other women. So perhaps we’ve 
 just been lied to all this time? It’s definitely something to look into for 
 sure.
 
 Now the calling and election made sure doctrine is to have Christ personally 
 visit you, it is determined that you will serve Christ no matter the cost. He 
 gives unto you the second comforter, that is the visitation of Christ.
 
 According to the caution above, it appears the calling and election being made 
 sure and the second anointing are two separate ordinances. For one you actually 
 see Christ the other you are just given the ability to be a God basically.
 
 We read about the calling and election being made sure in 2 Peter 1:10.
 
 \begin{displayquote}
 Wherefore the rather, brethren, give diligence to make your calling and 
 election sure; for if ye do these things, ye shall never fail:
 \end{displayquote}
 
 In the experiences I’ve read regarding the second anointing. Why is it only the 
 man is anointed a King and Priest to the Most High God? The woman is not 
 anointed along with her husband. Why is that? She performs the ordinance on him 
 after the apostle performs the ordinance and gives her husband a blessing. But 
 the husband does not in return give her a blessing and does not wash her feet. 
 Perhaps there are times when the woman is also anointed a Queen and Priestess, 
 the website ldsendowment.org has instructions for the entire process including 
 having the wife anointed. So perhaps in just the experiences I’ve read it was 
 only the husband. You may also read about this in a book by David John Buerger 
 titled "The Mysteries of Godliness."

 This doctrine is not taught in the church. It has been made known to us by 
 people who have experienced it. They have since fallen away from the church 
 because they have come to the conclusion that it is not true. But they have 
 brought forth the knowledge they have gained in this life with them thus far, 
 and that is well enough.

 There is also a paper titled “The Fulness of the Priesthood: The Second 
 Anointing in Latter-day Saint Theology and Practice” by David John Buerger. 
 According to that paper the second anointing was rather common to be had among 
 the saints. Prophets authorized the second anointing in the low thousands. It 
 makes one stop to think for a moment, why did it cease? Or well, I suppose it 
 dropped in number, not ceased. I would wager it takes place still today just 
 not as much as it used to.