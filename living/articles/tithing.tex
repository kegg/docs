\chapter{Tithing}

First an interesting quote regarding tithing. How it will eventually not be required
of the people.

\begin{displayquote}
I want to say to my brethren and sisters here this morning, that in my opinion there
never was a time when the members of the Church of Jesus Christ of Latter-day Saints
were living better lives, were more faithful and more diligent, than they are today.
We have various means of judging of this. One very accurate way of knowing is the
fact that the law of tithing is being observed. There never has been a time in the
history of the Church, I believe, when the law of tithing was observed more
universally and more honestly than it has been observed by the Latter-day Saints of
late. The tithes of the people during the year 1906, have surpassed the tithing of
any other year. This is a good indication that the Latter-day Saints are doing their
duty, that they have faith in the Gospel, that they are willing to keep the
commandments of God, and that they are working up to the line more faithfully perhaps
than ever before. I want to say another thing to you, and I do so by way of
congratulation, and that is, that we have, by the blessing of the Lord and the
faithfulness of the Saints in paying their tithing, been able to pay off our bonded
indebtedness. Today the Church of Jesus Christ of Latter-day Saints owes not a dollar
that it cannot pay at once. At last we are in a position that we can pay as we go. We
do not have to borrow any more, and we wont have to if the Latter-day Saints continue
to live their religion and observe this law of tithing, It is the law of revenue to
the Church.

Furthermore, I want to say to you, we may not be able to reach it right away, but we
expect to see the day when we will not have to ask you for one dollar of donation for
any purpose, except that which you volunteer to give of your own accord, because we
will have tithes sufficient in the storehouse of the Lord to pay everything that is
needful for the advancement of the kingdom of God. I want to live to see that day, if
the Lord will spare my life. It does not make any difference, though, so far as that
is concerned, whether I live or not. That is the true policy, the true purpose of the
Lord in the management of the affairs of His Church.

Before I sit down I would like to make another statement. Our enemies have been
publishing to the world that the Presidency of the Church and the leading officers
are consuming the tithes of the people. Now, I am going to tell you a little secret,
and it is this: there is not one of the general authorities in the Church that draws
one dollar from the tithes of the people for his own use. Well, you may say, how do
they live? I will give you the key: The Church helped to support in its infancy the
sugar industry in this country, and it has some means invested in that enterprise.
The Church helped to establish Z.C.M.I., and it has a little interest in that, and
in some other institutions which pay dividends. In other words, tithing funds were
invested in these institutions, which give employment to many, for which the
Trustee-in-Trust holds stock certificates, which are worth more today than what was
given for them; and the dividends from these investments more than pay for the
support of the general authorities of the Church. So we do not use one dollar of your
tithing. I thought I would like to tell you that much, so that when you hear men
talking about Joseph F. Smith and his associates consuming the tithes of the people
you can throw it back into their teeth that they do not use a dollar of the tithing
for their support. I would like our ``friends," if I might be permitted to use a
vulgar expression, to ``put that in their pipe and smoke it."
(Laughter.)\footnote{Seventy-Seventh Semi-Annual Conference, Joseph F. Smith, pp. 7-8}
\end{displayquote}

Some interesting thoughts put into there. I should be allowed to illustrate them.

1. Tithing was never meant to be lasting, the church required it at a certain period
of time in order to help build temples, get the church out of debt etc.

2. The money that pays for the prophets and apostles living expenses is from business
ventures. Which originated from tithing funds. So even though the prophets and
apostles do not get paid from tithing in these later days, it originally began with
tithing.

Another interesting tidbit about tithing, during the 70th Semi-Annual General
Conference of the Church, Lorenzo Snow stated the following:

\begin{displayquote}
``I plead with you in the name of the Lord, and I pray that every man, woman and 
child ... shall pay one tenth of their income as a tithing."\footnote{
Teachings of Presidents of the Church: Lorenzo Snow (2011)
}
\end{displayquote}

Wait, what's with the elipses? Oh that quote is from the Teachings of Presidents of
the Church: Lorenzo Snow. Here's the quote in its entirety from the conference
report, I'll add emphasis on the important bit:

\begin{displayquote}
``I plead with you in the name of the Lord, and I pray that every man, woman and child 
\textbf{who has means} shall pay one tenth of their income as a tithing."\footnote{
In Conference Report, Oct. 1899, 28
}
\end{displayquote}

Why did they leave out the words ``who has means" in the teachings of the presidents
book? Leaving those words out, changes the meaning of the sentence. Elipses are so
annoying at times. You have to hunt down the entire quote as it was originally
intended to be said. If not? Then don't be quoting it out of context.