\chapter{Nephi Kills Laben}

Question: When is it okay to kill another person?

Answer:

\begin{displayquote}
Thou shalt not kill.\footnote{Exodus 20:13}
\end{displayquote}

\begin{displayquote}
And again, the Lord God hath commanded that men should not 
murder.\footnote{2 Nephi 26:32}
\end{displayquote}

\begin{displayquote}
And now, behold, I speak unto the church. Thou shalt not kill; 
and he that kills shall not have forgiveness in this world, nor in the 
world to come.\footnote{D\&C 42:18}
\end{displayquote}

So, from the standard works, we have it that killing is bad. Murder is bad and evil
in the sight of God.

Yet, Nephi murdered Laben. Not only did he murder Laben, he hacked off his head.

If God can simply smite someone down\footnote{Abraham 1:20}, or cause a deep sleep 
to come upon them\footnote{Genesis 2:21}, why was it necessary for Nephi to kill 
Laben? Even Alma the Younger and the sons of Mosiah were shook by an Angel of the 
Lord. Alma the Younger even went into a coma like state as he was 
converted.\footnote{Mosiah 27:19} Why didn't the Lord just convert Laben in a like 
manner?

We are told he was ``constrained by the Spirit that [he] should kill 
Laban."\footnote{1 Nephi 4:10} Nephi admits he had never taken
another man's life. He resisted and resisted, and the spirit told Nephi ``Slay him,
for the Lord hath delivered him into thy hands."\footnote{1 Nephi 4:12}

So, Nephi chopped off Laben's head and went on his way. All because: ``It is better
that one man should perish than that a nation should dwindle and perish in
unbelief."\footnote{1 Nephi 4:13}

All of this was so they could get the plates of brass. The plates of brass contained
the writings of the Old Testament and the writings of Isaiah. Which, as discussed in
another article, Joseph Smith didn't even use the plates when translating them, so
was it necessary for Nephi to even get the plates of brass?

If the plates of brass contained the five books of Moses one would wonder if it is
the same story of Adam and Eve found in the Pearl of Great Price (Joseph Smith's
translation of the Bible). In Moses 6, it clearly states Jesus Christ's name. If
Nephi did have those writings, then he should have known Christ's name. Yet it
had to be told by an angel in 2nd Nephi.\footnote{2nd Nephi 10}

The Pentateuch wasn't formed until 400 BCE, which is 200 years after Lehi left
Jerusalem. The Pentateuch is the five books of Moses.\footnote{Pentateuch, LDS
Bible Dictionary}
