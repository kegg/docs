\chapter{Apostasy}

According to the Church's website, Apostasy is ``when individuals or groups of people
turn away from the principles of the gospel."\footnote{Apostasy, LDS.org}

When a person is in apostasy, they are requried to undergo a Church Disciplinary
Council. There are guidelines when such a council is mandatory. It also defines
Apostasy:

\begin{displayquote}
\textbf{Apostasy}\footnote{Handbook 1, 6.7.3}

As used here, \textit{apostasy} refers to members who:

\begin{enumerate}
\item Repeatedly act in clear, open, and deliberate public opposition to the Church 
  or its leaders.

\item Persist in teaching as Church doctrine information that is not Church doctrine 
  after they have been corrected by their bishop or a higher authority.

\item Continue to follow the teachings of apostate sects (such as those that advocate 
  plural marriage) after being corrected by their bishop or a higher authority.

\item Are in a same-gender marriage.

\item Formally join another church and advocate its teachings.
\end{enumerate}
\end{displayquote}

What concerns me the most is when a good moral person is standing up for something
they believe in, and is morally sound, that they get placed under church discipline.
They are considered an apostate and are treated as such.

Is this how God want's his church to be run? Is this how Jesus wants his church to be
run?

I am confused.