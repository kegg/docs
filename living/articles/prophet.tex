\chapter{The Prophet}

It has been said, when the prophet speaks the debate is over.\footnote{
Ensign, Nov. 1978, p. 108
}

Ezra Taft Benson, then a member of the Quorum of the Twelve Apostles, said the
following:

\begin{displayquote}
The prophet does not have to say ``Thus saith the Lord" to give us
scripture.\footnote{Fourteen Fundamentals in FOllowing The Prophet, Ezra Taft 
Benson, 1980}
\end{displayquote}

He then went on to Quote Brigham Young, who said:

\begin{displayquote}
I have never yet preached a sermon and sent it out to the children of men, that 
they may not call scripture.\footnote{Journal of Discourses, 13:95.}
\end{displayquote}

I wonder if Benson even read some of the things Brigham had taught over the years?
He's quoting from the Journal of Discourses, so he should have some knowledge of
things that Brigham actually taught as doctrine. One would think this would be the
case.

To be taught to always follow the prophet no matter what, it's blind obedience.
There's no other way of saying what it is. That's exactly what it is. When a group is
told to not think about it, to simply follow a living prophet and what he says goes?
It feels much like being led down a path where you have no say in anything. If a
person even thinks about disagreeing with that prophet? They are in danger of being
called an apostate.

In later years, we are told that prophets are not always speaking as a prophet.
Sometimes they are speaking as a man. It becomes difficult to understand when a
prophet is talking for God and when he is talking as a man. Conflicting thoughts and
speaches, quotes abound, some of which have been discussed already.

Let's take a look at some quotes where the prophet spoke, shall we? I'll let you
decide if they were speaking as a man or from God.

\begin{displayquote}
The attitude of the Church with reference to Negroes remains as it has always 
stood. It is not a matter of the declaration of a policy but of direct commandment 
from the Lord, on which is founded the doctrine of the Church from the days of 
its organization, to the effect that Negroes may become members of the Church 
but that they are not entitled to the priesthood at the present time.\footnote{
The First Presidency on the Negro Question, 17 Aug. 1949.
}
\end{displayquote}

\begin{displayquote}
Shall I tell you the law of God in regard to the African race? If the white man 
who belongs to the chosen seed mixes his blood with the seed of Cain, the penalty, 
under the law of God, is death on the spot. This will always be so. The nations of 
the earth have transgressed every law that God has given, they have changed the 
ordinances and broken every covenant made with the fathers, and they are like a 
hungry man that dreameth that he eateth, and he awaketh and behold he is
empty.\footnote{Prophet Brigham Young, Journal of Discourses, v. 10, p. 110}
\end{displayquote}

\begin{displayquote}
I [am] opposed to hanging, even if a man kill another, I will shoot him, or cut 
off his head, spill his blood on the ground, and let the smoke thereof ascend up t
o God; and if ever I have the privilege of making a law on that subject, 
I will have it so.\footnote{Prophet Joseph Smith, Jr., History of the Church, v. 
5, p. 296, 1949}
\end{displayquote}

\begin{displayquote}
Will you love your brothers and sisters likewise, when they have committed a sin 
that cannot be atoned for without the shedding of their blood? Will you love that 
man or woman well enough to shed their blood? That is what Jesus Christ
meant.\footnote{Prophet Brigham Young, Deseret News, April 16, 1856}
\end{displayquote}

\begin{displayquote}
Suppose you found your brother in bed with your wife, and put a javelin through 
both of them. You would be justified, and they would atone for their sins, and be 
received into the Kingdom of God. I would at once do so, in such a case; and under 
the circumstances, I have no wife whom I love so well that I would not put a 
javelin through her heart, and I would do it with clean hands.... There is not a 
man or woman, who violates the covenants made with their God, that will not be 
required to pay the debt. The blood of Christ will never wipe that out, your own 
blood must atone for it.\footnote{Prophet Brigham Young, Journal of Discourses, v. 
1, pp. 108-109}
\end{displayquote}

\begin{displayquote}
Joseph Smith taught that there were certain sins so grievous that man may commit, 
that they will place the transgressors beyond the power of the atonement of Christ. 
If these offenses are committed, then the blood of Christ will not cleanse them 
from their sins even though they repent. Therefore their only hope is to have 
their blood shed to atone, as far as possible, in their behalf. This is scriptural 
doctrine, and is taught in all the standard works of the Church.\footnote{
Prophet Joseph Fielding Smith, Doctrines of Salvation, v. 1, pp. 135-136, 1954
}
\end{displayquote}

There are some interesting ``doctrine" taught here. (They claim it's doctrine so
is it not?) Were these men teaching as prophets or their own thoughts? They claim
it's doctrine, and that's how things are. So what is a person to believe regarding
it?

\section{The Living Prophet is more important than the scriptures.}

It has been said that basically the current or modern prophet is more important than
the scriptures.

\begin{displayquote}
We are admonished to ``seek out of the best books words of wisdom" (D\&C 88:118). 
Surely these books must include the scriptures. Alongside them must be the words of 
the Presidents of the Church. The Lord said of the President of the Church, ``His 
word ye shall receive, as if from mine own mouth" (D\&C 21:5). These books make up 
what has been referred to as ``the Lord’s library"—namely the standard works and 
the various volumes that contain the words of the different Presidents of the Church. 
Of the latter volumes, that which would be of greatest importance to you would be 
the words of the current President of the Church, for his words are directed to our 
day and our needs. (Teachings of Ezra Taft Benson, p.137-138)
\end{displayquote}

\begin{displayquote}
I will refer to a certain meeting I attended in the town of Kirtland in my early 
days. At that meeting some remarks were made that have been made here today, with 
regard to the living oracles and with regard to the written word of God. The same 
principle was presented, although not as extensively as it has been here, when a 
leading man in the Church got up and talked upon the subject, and said: 
``You have got the word of God before you here in the Bible, Book of Mormon, and 
Doctrine and Covenants; you have the written word of God, and you who give 
revelations should give revelations according to those books, as what is written in 
those books is the word of God. We should confine ourselves to them." When he 
concluded, Brother Joseph turned to Brother Brigham Young and said, ``Brother 
Brigham I want you to take the stand and tell us your views with regard to the 
written oracles and the written word of God." Brother Brigham took the stand, and 
he took the Bible, and laid it down; he took the Book of Mormon, and laid it down; 
and he took the Book of Doctrine and Covenants, and laid it down before him, and he 
said: ``There is the written word of God to us, concerning the work of God from the 
beginning of the world, almost, to our day." ``And now," said he, ``when compared 
with the living oracles those books are nothing to me; those books do not convey 
the word of God direct to us now, as do the words of a Prophet or a man bearing the 
Holy Priesthood in our day and generation. I would rather have the living oracles 
than all the [p.23]writing in the books." That was the course he pursued. When he was 
through, Brother Joseph said to the congregation: ``Brother Brigham has told you the
word of the Lord, and he has told you the truth."  (Conference Report, 
  October 1897, p.22)
\end{displayquote}