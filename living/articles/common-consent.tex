\chapter{Law of Common Consent}

If a person doesn't support a policy that was given and is said to be revelation, how
exactly does that get handeled? I dare not ask, because there are a few policies
which I believe they are actually policies but not revelation.

Putting something in a handbook without telling the church is not revelation. That is
a policy change. The handbook is not scripture. Shouldn't revelation be considered
scripture? There is a line between scripture and that which is drafted by lawyers and
placed in a handbook. It even states it's a ``Policy".

\begin{displayquote}
Policies on Ordinances for Children of a Parent Living in a Same-Gender
Relationship\footnote{Changes to LDS Handbook 1- Document 2 Revised 11-3-2015.pdf}
\end{displayquote}

Since when has a policy become scripture or a revelation? Are not revelations from
the Lord considered scripture?\footnote{\textbf{Revelation:} Communication from God 
to His children on earth. Revelation may come through the Light of Christ and the 
Holy Ghost by way of inspiration, visions, dreams, or visits by angels. Revelation 
provides guidance that can lead the faithful to eternal salvation in the celestial 
kingdom.

The Lord reveals His work to His prophets and confirms to believers that the 
revelations to the prophets are true (Amos 3:7). Through revelation, the Lord 
provides individual guidance for every person who seeks it and who has faith, 
repents, and is obedient to the gospel of Jesus Christ. ``The Holy Ghost is a 
revelator," said Joseph Smith, and ``no man can receive the Holy Ghost without 
receiving revelations."

In the Lord’s Church, the First Presidency and the Quorum of the Twelve Apostles are 
prophets, seers, and revelators to the Church and to the world. The President of the 
Church is the only one whom the Lord has authorized to receive revelation for the 
Church (D\&C 28:2–7). Every person may receive personal revelation for his own 
benefit.

By every word that proceedeth out of the mouth of the Lord doth man live, Deut. 8:3
(Matt. 4:4; D\&C 98:11). [LDS.org]}

So, what becomes of it all? If things in the handbook are considered revelation, but
not scripture. It becomes confusing, and goes against the law of common consent. The
law of common consent is stated as follows:

\begin{displayquote}
Not only are Church officers sustained by common consent, but this same principle 
operates for policies, major decisions, acceptance of new scripture, and other things 
that affect the lives of the Saints (see D\&C 26:2).\footnote{``Section 26, The Law 
of Common Consent," Doctrine and Covenants Student Manual (2002), 54}
\end{displayquote}

As far as I am aware, there hsan't been a sustaining vote on a revelation since the
1978 announcement of the ban on the blacks in General Conference.\footnote{
``Revelation on Priesthood Accepted, Church Officers Sustained," October 1978 General
Conference, LDS.org
}

It would be interesting to list policies and revelations which haven't been voted on,
I'll start with the most recent and work my way backwards? Maybe? Who knows how far I
can go with it. These have been stated as being revelation.

\begin{enumerate}
\item No longer referring to the church as the Mormon church (2018)
\item Discontinuing Home Teaching, adopting Ministering (2018)
\item Combining Elders Quorum and High Priests Group into one Elders Quroum (2018)
\item Children of Same-Gender Relationships (2015)
\item Lowering of Mission Ages (2012)
\end{enumerate}

I heard it considered that because a group sustains the prophet, they don't have to
live by the law of common consent. That because they sustain the prophet, they
automatically accept whatever revelation that comes and is announced. (For the
  record, the Children of Same-Gender Relationships policy, wasn't revealed to
  the membership of the church until after it was leaked. It was snuck in Handbook
  1.)

Then Apostle Russel M. Nelson had the following to say regarding the policy:

\begin{displayquote}
We sustain 15 men who are ordained as prophets, seers, and revelators. When a thorny 
problem arises-and they only seem to get thornier each day—these 15 men wrestle with 
the issue, trying to see all the ramifications of various courses of action, and they 
diligently seek to hear the voice of the Lord. After fasting, praying, studying, 
pondering, and counseling with my Brethren about weighty matters, it is not unusual 
for me to be awakened during the night with further impressions about issues with 
which we are concerned. And my Brethren have the same experience.

The First Presidency and Quorum of the Twelve Apostles counsel together and share 
all the Lord has directed us to understand and to feel individually and collectively. 
And then we watch the Lord move upon the President of the Church to proclaim the 
Lord's will.

This prophetic process was followed in 2012 with the change in minimum age for 
missionaries and again with the recent additions to the Church's handbook, 
consequent to the legalization of same-sex marriage in some countries. Filled with 
compassion for all, and especially for the children, we wrestled at length to 
understand the Lord’s will in this matter. Ever mindful of God's plan of salvation 
and of His hope for eternal life for each of His children, we considered countless 
permutations and combinations of possible scenarios that could arise. We met 
repeatedly in the temple in fasting and prayer and sought further direction and 
inspiration. And then, when the Lord inspired His prophet, President Thomas S. 
Monson, to declare the mind of the Lord and the will of the Lord, each of us during 
that sacred moment felt a spiritual confirmation. It was our privilege as Apostles 
to sustain what had been revealed to President Monson. Revelation from the Lord to 
His servants is a sacred process, and so is your privilege of receiving personal 
revelation.\footnote{Becoming True Millennials, Russel M. Nelson, January 10, 2016 
Devotional}
\end{displayquote}