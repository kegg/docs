\chapter{Baptismal Prayers}

Over the years, there have been a few variations on baptismal prayers. These
are found in the Book of Mormon and in the Doctrine and Covenants. Why change
those? They all have a common theme, that you are required to state authority
from God. But if that's the case, then why do we have to cite a baptismal
prayer so specifically in today's time?

Here are three different versions, first two are from the Book of Mormon, the
third is from the Doctrine and Covenants:

\begin{displayquote}
Having authority given me of Jesus Christ, I baptize you in the name of the 
Father, and of the Son, and of the Holy Ghost. Amen.\footnote{3 Nephi 11:25}
\end{displayquote}

\begin{displayquote}
Helam, I baptize thee, having authority from the Almighty God, 
as a testimony that ye have entered into a covenant to serve him 
until you are dead as to the mortal body; and may the Spirit of 
the Lord be poured out upon you; and may he grant unto you 
eternal life, through the redemption of Christ, 
whom he has prepared from the foundation of the world.\footnote{Mosiah 18:13}
\end{displayquote}

\begin{displayquote}
Having been commissioned of Jesus Christ, I baptize you in the name of the 
Father, and of the Son, and of the Holy Ghost. Amen.\footnote{D\&C 20:73}
\end{displayquote}

I find it interesting that there ins't any specific wording for baptismal
prayers in the New Testament of the Holy Bible. We find talk of baptism and
that it is necessary to repent etc. but no specific prayers:

\begin{displayquote}
Then Peter said unto them, Repent, and be baptized every one of you in the 
name of Jesus Christ for the remission of sins, and ye shall 
receive the gift of the Holy Ghost.\footnote{Acts 2:38}
\end{displayquote}

So it's important to be baptized in the name of Jesus. There aren't specific
words to be taken into account. Obviously God respects and expects authority
be used in the baptizing, but that seems about it.

There is record in Matthew that people are to go to all the world,
``baptizing them in
the name of the Father, 
and of the Son, and of the Holy Ghost"\footnote{Matthew 28:19}

There is talk about rising up out of the water as it 
were,\footnote{Acts 8:36-39} and that we are buried in 
baptism;\footnote{Romans 6:4} which would seem to indicate the method
by with baptism is accomplished.

Yet no specific wording, no that came much later with Moroni.

If people were not baptized with the same wording thoughout the years, is their
baptism accepted by God? Is it the spirit of the law and not the letter of the
law?