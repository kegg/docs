\chapter{Race and the Eternal Salvation Ban}

Long ago, the LDS Church stated that the negro race weren't allowed to hold the 
Holy Priesthood of God. They weren't able to attend the temple either. This 
lasted for several years. Then change came about, and in 1978 a revelation was 
passed down that lifted this ban.

Before the change, presidents and apostles of the church had no issue stating in 
no uncertain terms that the ban was of God. It was god's doing, the Lord put the 
ban in place and it was His purpose for doing so. That it was doctrine.

\begin{displayquote}
Ham, through Egyptus, continued the curse which was placed upon the seed of 
Cain. Because of that curse this dark race was separated and isolated from all 
the rest of Adam's posterity before the flood, and since that time the same 
condition has continued, and they have been `despised among all people.' 
This \textbf{doctrine} did not originate with President Brigham Young but was 
taught by the Prophet Joseph Smith ... we all know it is due to his teachings 
that the negro today is barred from the 
Priesthood.\footnote{The Way to Perfection, pages 110-111}
\end{displayquote}

However, according to the Gospel Topic Essays, we learn:

\begin{displayquote}
Over time, Church leaders and members advanced many theories to explain the 
priesthood and temple restrictions. None of these explanations is accepted 
today as the official doctrine of the 
Church.\footnote{Race and the Priesthood, LDS.org}
\end{displayquote}

Others, like John Taylor, taught more ... unsettling things:

\begin{displayquote}
And after the flood we are told that the curse that had been pronounced upon 
Cain was continued through Ham's wife, as he had married a wife of that seed. 
And why did it pass through the flood? because it was necessary that the devil 
should have a representation upon the earth as well as 
God ...\footnote{Journal of Discourses, 22:304}
\end{displayquote}

This was focused on more than once:

\begin{displayquote}
Why is it, in fact, that we should have a devil? Why did the Lord not kill 
him long ago? Because he could not do without him. He needed the devil and a 
great many of those who do his bidding to keep men straight, that we may learn 
to place our dependence on God, and trust in Him, and to observe his laws and 
keep his commandments. When he destroyed the inhabitants of the antediluvian 
world, he suffered a descendant of Cain to come through the flood in order 
that he might be properly represented upon the 
earth.\footnote{Journal of Discourses, 23:336}
\end{displayquote}

Then there were these quotes:

\begin{displayquote}
Shall I tell you the law of God in regard to the African race? If the white 
man who belongs to the chosen seed mixes his blood with the seed of Cain, 
the penalty, under the law of God, is death on the spot. This will 
always be so.\footnote{Brigham Young, Journal of Discourses, Vol 10, page 110}
\end{displayquote}

Some taught that it was of God and was the Lord's doing:

\begin{displayquote}
Negroes in this life are denied the Priesthood; under no circumstances can 
they hold this delegation of authority from the Almighty. (Abra. 1:20-27.) 
The gospel message of salvation is not carried affirmatively to them... 
negroes are not equal with other races where the receipt of certain 
spiritual blessings are concerned, particularly the priesthood and 
the temple blessings that flow there from, but this inequality is 
not of man's origin. It is the Lord's doing, is based on his eternal 
laws of justice, and grows out of the lack of Spiritual valiance of those 
concerned in their first estate.\footnote{Mormon Doctrine, 1966, pp. 527-528}
\end{displayquote}

The church even presented a proclamtion out to the world about it.

\begin{displayquote}
The attitude of the Church with reference to Negroes remains as it has always
stood. It is not a matter of the declaration of a policy but of direct commandment
from the Lord, on which is founded the doctrine of the Church from the days of its 
organization, to the effect that Negroes may become members of the Church but
that they are not entitled to the priesthood at the present time.
\footnote(Statement of the First Presidency of the Church of Jesus Christ of 
	Latter-day Saints, August 17, 1949)
\end{displayquote}

So it was a commandment from the Lord and doctrine at the time. Interesting.

There are several more I could include in this paper, but I believe these
are sufficient for now.

Personally, reading such quotes turns my stomach. I do not understand how a 
prophet of God could speak like that. If we are truely to love our brothers
and sisters as Christ taught, one would think that these teachings wouldn't
have occurred.

After the ban, they said it was folklore. The reasons for doing so was because 
of man.

How is it one generation of prophets and apostles can call an older generation 
false on their teachings? Teachings people followed because they were following 
the prophet?

Brigham Young stated that he was afraid people would have too much faith in the 
presidency of the church and him as a prophet that they wouldn't ask God if 
something was right.\footnote{I am more afraid that this people have so much 
confidence in their leaders that they will not inquire for themselves of God 
whether they are lead by him. [Brigham Young, (12 January 1862) Journal of 
Discourses 9:150]}

Well, if you were all for a black person getting the priesthood, or questioned 
your leaders about it because you felt that was the correct course of 
action...you were facing excommunication.

So change can come, but it can also come at quite a price.

I think it is okay to ask this. If the ban wasn't of God as church leaders are 
now saying, then why did God allow it? Why would God allow such a thing to take 
place? If it was indeed ``folklore", one would think God wouldn't allow prophets 
and apostles of the church to allow people to think the negro would never 
receive the priesthood and temple ordinances.

There are so many quotes on the matter it sickens me to think about it.

Even after the ban, this ``folklore" was still taught on the lds.org 
website that it was from God as far forward as 2010.

\begin{displayquote}
Ever  since  biblical  times,  the  Lord  has  designated  through  His  
prophets  who  could  receive  the priesthood  and  other  blessings  of  the  
gospel.  Among  the  tribes  of  Israel,  for  example,  only  men  of  
the tribe  of  Levi  were  given  the  priesthood  and  allowed  to  
officiate  in  certain  ordinances.  Likewise,  during  the Savior's  
earthly  ministry,  gospel  blessings  were  restricted  to  the  Jews.  
Only  after  a  revelation  to  the Apostle  Peter  were  the  gospel  and  
priesthood  extended  to  others  
(see  Acts  10:1-33;  
14:23;  15:6–8).\footnote{Priesthood  Ordination  before  1978, lds.org}
\end{displayquote}

But it is all cleared up by one remark by an apostle. Bruce R. McConkie:

\begin{displayquote}
Forget everything that I have said, or what President Brigham Young or 
President George Q. Cannon or whomsoever has said in days past that is contrary 
to the present revelation. We spoke with a limited understanding and without the 
light and knowledge that now has come into the world.\footnote{All Are Alike 
Unto God, Bruce R. McConkie, Aug 18, 1978}
\end{displayquote}

In The Deseret News, we find a quote from Jeffry R. Holland:

\begin{displayquote}
Likewise, the current leadership of the church has spoken on the need to 
abandon the racist teachings that long circulated within Mormonism 
regarding the ban. Elder Jeffery R. Holland, a current member of the 
Council of the Twelve, recently said in a public interview 
``One clear-cut position is that the folklore must never be perpetuated...
I think almost all of (these teachings) were inadequate and/or 
wrong."\footnote{Deseret News, Race, folklore and Mormon doctrine, 
Nathan B. Oman, February 29, 2012}
\end{displayquote}

If change can be simply accepted based on a revelation from God then that is 
good right? Why did it take a revelation to change policy? The church claims it 
was a policy not doctrine. Even though it was taught as doctrine throughout the 
course of history.

Do the lines between doctrine and policy blur at times? Perhaps more change?

There are scriptures that reference to the people's skin being turned dark due 
to sin or not following God's will while on the Earth. Cain was the first man to 
go dark because of murder. A mark of darkness was placed upon man in the event 
that anyone would come across him.\footnote{And the Lord said unto him, 
Therefore whosoever slayeth Cain, vengeance shall be taken on him sevenfold. 
And the Lord set a mark upon Cain, lest any finding him should kill 
him.[Genesis 4:15]}

In the Book of Mormon we learn about the Lamanites and the Nephites. The 
Lamanites had the dark skin:

\begin{displayquote}
And the skins of the Lamanites were dark, according to the mark which was set 
upon their fathers, which was a curse upon them because of their transgression 
and their rebellion against their brethren, who consisted of Nephi, Jacob, and 
Joseph, and Sam, who were just and holy men.\footnote{Alma 3:6}
\end{displayquote}

God says that the cursing is so the wicked people wouldn't be enticing to those
who followed the commandments of God.

\begin{displayquote}
And he had caused the cursing to come upon them, yea, even a sore cursing, 
because of their iniquity. For behold, they had hardened their hearts against 
him, that they had become like unto a flint; wherefore, as they were white, 
and exceedingly fair and delightsome, that they might not be enticing unto my 
people the Lord God did cause a skin of blackness to come upon 
them.\footnote{2 Nephi 5:21}
\end{displayquote}

Before 1978, there was certain things taught. One of those teachings was that 
the dark skinned people were less valiant in the pre-existence. This has been
shot down. There were no fence sitters in the pre-existence in the war in 
heaven. Either you chose Jesus or you chose Lucifer.\footnote{Apostle Joseph 
Fielding Smith, for example, wrote in 1907 that the belief was ``quite general" 
among Mormons that ``the Negro race has been cursed for taking a neutral 
position in that great contest." Yet this belief, he admitted, ``is not the 
official position of the Church, [and is] merely the opinion of men." 
Joseph Fielding Smith to Alfred M. Nelson, Jan. 31, 1907, 
Church History Library, Salt Lake City.}

Now you'll notice I called this an ``Eternal Salvation Ban" not simplay a 
``Priesthood Ban" as the church tends to simplify it. No, it's more than that.
It was a temple ban. People of color weren't able to be sealed to their loved
ones, which is one of the main points of LDS Doctrine. The idea of eternal
families.

Feels like a slap to the face of those wanting to be sealed to their spouses,
children, parents, loved ones etc. If you claim to have revelation from God and
part of that is that the whole human race has the ability to be together 
forever, why would God allow for man to withold that from his children?

\section{1978 Revelation}

Then in 1978, there was a change of policy.

Now, the church considers it a revelation. But in an interview with the apostle
LeGrand Richards, it sounds quite different.

\begin{displayquote}
WALTERS: Now when President Kimball read this little announcement or paper, 
was that the same thing that was released to the press?

RICHARDS: Yes.

WALTERS: There wasn't a special document as a ``revelation", that he had and 
wrote down?

RICHARDS: We discussed it in our meeting. What else should we say besides 
that announcement? And we decided that was sufficient; that no more 
needed to be said.\footnote{Interview with Apostle LeGrand Richards,
By Wesley P. Walters and Chris Vlachos, 16th August 1978, Church Office Building
(Recorded on Cassette)}
\end{displayquote}

There was no ``Thus saith the Lord" in the Official Declaration 2. So I question
you, dear reader, was it a revelation? I dare say it wasn't. I dare say it was
a policy change. I dare say what was once taught as doctrine and taught as it
was from God was changed by the pressures and will of man.

Speaking of Official Declaration 2, here is the text in its entirety.

\begin{displayquote}
To Whom It May Concern:

On September 30, 1978, at the 148th Semiannual General Conference of The 
Church of Jesus Christ of Latter-day Saints, the following was presented by 
President N. Eldon Tanner, First Counselor in the First Presidency of the 
Church:

In early June of this year, the First Presidency announced that a revelation 
had been received by President Spencer W. Kimball extending priesthood and 
temple blessings to all worthy male members of the Church. President Kimball 
has asked that I advise the conference that after he had received this 
revelation, which came to him after extended meditation and prayer in the 
sacred rooms of the holy temple, he presented it to his counselors, who 
accepted it and approved it. It was then presented to the Quorum of the 
Twelve Apostles, who unanimously approved it, and was subsequently presented 
to all other General Authorities, who likewise approved it unanimously.

President Kimball has asked that I now read this letter:

June 8, 1978

To all general and local priesthood officers of The Church of Jesus Christ 
of Latter-day Saints throughout the world:

Dear Brethren:

As we have witnessed the expansion of the work of the Lord over the earth, we 
have been grateful that people of many nations have responded to the message 
of the restored gospel, and have joined the Church in ever-increasing numbers. 
This, in turn, has inspired us with a desire to extend to every worthy member 
of the Church all of the privileges and blessings which the gospel affords.

Aware of the promises made by the prophets and presidents of the Church who have 
preceded us that at some time, in God's eternal plan, all of our brethren who 
are worthy may receive the priesthood, and witnessing the faithfulness of those 
from whom the priesthood has been withheld, we have pleaded long and earnestly 
in behalf of these, our faithful brethren, spending many hours in the Upper Room 
of the Temple supplicating the Lord for divine guidance.

He has heard our prayers, and by revelation has confirmed that the long-promised 
day has come when every faithful, worthy man in the Church may receive the holy 
priesthood, with power to exercise its divine authority, and enjoy with his 
loved ones every blessing that flows there from, including the blessings of the 
temple. Accordingly, all worthy male members of the Church may be ordained to 
the priesthood without regard for race or color. Priesthood leaders are 
instructed to follow the policy of carefully interviewing all candidates for 
ordination to either the Aaronic or the Melchizedek Priesthood to insure that 
they meet the established standards for worthiness.

We declare with soberness that the Lord has now made known his will for the 
blessing of all his children throughout the earth who will hearken to the 
voice of his authorized servants, and prepare themselves to receive every 
blessing of the gospel.

Sincerely yours,

SPENCER W. KIMBALL

N. ELDON TANNER

MARION G. ROMNEY

The First Presidency

Recognizing Spencer W. Kimball as the prophet, seer, and revelator, and 
president of The Church of Jesus Christ of Latter-day Saints, it is proposed 
that we as a constituent assembly accept this revelation as the word and 
will of the Lord. All in favor please signify by raising your right hand. 
Any opposed by the same sign.

The vote to sustain the foregoing motion was unanimous in the affirmative.

Salt Lake City, Utah, September 30, 1978.\footnote{Official Declaration 2, 
Doctrine and Covenants}
\end{displayquote}

It is a wonderful thing that this ban was lifted. It is a shame it ever was in
place to begin with. Imagine all of those years of racism and hatred that could
have been done without. People believed God spoke and they followed Him. The
prophet led them and he couldn't be wrong...even when he was saying that those
under the ban would never receive the priesthood in this life.

If they were speaking as men, which I truely hope they were, why would God allow
such a thing? Why would He allow such teachings to go on for so many years? I 
ask it all again. Why?

There are many things in this life that don't add up or make sense. I suppose
this is one of them. To understand it in another life, to have to wait to be 
able to understand it in another life? Why would that be? It would seem with
the changing narrative, dismissing those who have spoken ``as prophets of God",
seems to downplay it all. The church doesn't want to come off as racist. That
is understandable. But instead of brushing it under a rug, why not apologize?

Was there ever a full formal apology regarding it? Or was this new ``revelation"
simply all there was to make things better? It feels like they put a band-aid
over a wound simply to let it heal and go away eventually.

The interesting thing about history, it doesn't just go away. Those teachings
of former prophets are still around. With the internet and this day in age,
those teachings will never be lost. No matter how much people wish it would go
away, it will never be lost. People will always be able to find it, research it,
and learn what happened and form an opinion on it; after they have read all of
the facts.

I know people always say, well that's old church history. That has nothing to do
with current prophets and apostles. The 70s weren't that long ago. let's not
forget that.

Okay, you want something more today? What about 2018? Is that more current
enough?

During the Be One celebration of 40 years since the Race and Eternal Salvation
ban was lifted, here's what one of the current general authorities said about
it:

\begin{displayquote}
I observed the pain and frustration experienced by those who suffered these 
restrictions and those who criticized them and sought for reasons. I studied the 
reasons then being given and could not feel confirmation of the truth of any of 
them. As part of my prayerful study, I learned that, in general, the Lord rarely 
gives reasons for the commandments and directions He gives to His servants. I 
determined to be loyal to our prophetic leaders and to pray — as promised from 
the beginning of these restrictions — that the day would come when all would 
enjoy the blessings of priesthood and temple.\footnote{
https://www.ldschurchnews.com/latest/2018-06-01/president-oaks-full-remarks-from-the-lds-churchs-be-one-celebration-47280
}
\end{displayquote}

So, there you have it. A current leader admitting that he couldn't find truth in
teachings of previous leaders. He also said that it was a commandment from the
Lord that the ban be there, not simply racism in the church. A commandment
from the Lord. He buckled down and followed the prophet...even though it was
morally wrong to do so. To follow blindly is not wise, to learn and seek for
yourself and then to rise up if needed is better. If this is what we are taught,
then I want no part of it.

The church keeps saying that God is no respector of persons, yet it was God who
instituted the ban. It was a commandment from God that those of African descent
could not receive the priesthood or blessings of the temple. If God is no
respector of persons, and all of His children are equal in His eyes, why did He
command such a ban to be made? Why did it take a ``revelation" to make the ban
null and void? Couldn't someone in the first presidency simply said, this is
wrong and we must put a stop to it? I dare say they could have. I even dare say
they should have!

Again, God tells us that we should not be commanded in all things. Surley this
is one of thise things we shouldn't have been commanded to do or not do! I
believe a revelation wasn't necessary in any of this. The scriptures say to love
your neighbor as yourself, that is one of the great commandments. Because of the
Racial Ban, there was no love as far as race was concerned. There are numerous
quotes about racism in the early church. I have listed them above already. There
are more than those of course but that sample is more than enough.

\section{Lawsuits}

\begin{displayquote}
In July 1974, the NAACP filed a suit against the Boy Scouts of America on the grounds 
that in LDS troops ... the deacons quorum president was automatically the patrol 
leader, meaning that an African-American Scout could not gain patrol leader 
experience. When the church realized the inappropriateness of such restriction ... 
it dropped the policy and the court dismissed the suit. President [Spencer] Kimball 
had been subpoenaed to appear for deposition and bring ``all church records and 
writings concerning the policy and position of the church regarding blacks." He 
felt greatly relieved at avoiding that burden and the inevitable adverse publicity.
\footnote{Salt Lake Tribune, LDS President Kimball -- now you can read the rest of the 
story, Peggy Fletcher Stack, January 29, 2010}
\end{displayquote}

This incident was included in the exommunication of Byron Marchant who cast a vote
against a leader of the church. He was the leader of such a scout troop. But because
he opposed church policy regarding blacks and the priesthood openly, he was
excommunicated.\footnote{See The Daily Reporter (Dover, Ohio) 15 Oct 1977}

\section{White and Delightsome}

From the 1940s to the year 2000, there was an Indian Placement Program in the church.
It was believed by some that Native Americans in this program would become white and
delightsome like the white church membrers they were placed with. Spencer W. Kimball
said the following:

\begin{displayquote}
The day of the Lamanites is nigh. For years they have been growing delightsome, and
they are now becoming white and delightsome, as they were promised. In this picture
of the twenty Lamanite missionaries, fifteen of the twenty were as light as Anglos;
five were darker but equally delightsome. The children in the home placement program
in Utah are often lighter than their brothers and sisters in the hogans on the
reservation.... At one meeting a father and mother and their sixteen-year-old daughter
were present, the little member girl-sixteen sitting between the dark father and
mother, and it was evident she was several shades lighter than her parents on the
same reservation, in the same Hogan, subject to the same sun and wind and weather.
There was the doctor in a Utah city who for two years had had an Indian boy in his
home who stated that he was some shades lighter than the younger brother just coming
into the program from the reservation. These young members of the Church are changing
to whiteness and delightsomeness. One white elder jokingly said that he and his
companion were donating blood regularly to the hospital in the hope that the process
might be accelerated.\footnote{Prophet Spencer W. Kimball, 
General Conference, Oct. 1960}
\end{displayquote}

The program wasn't successful and the church discontinued it.