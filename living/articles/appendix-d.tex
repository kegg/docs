\chapter{ON MARRIAGE}

According to the custom of all civilized nations, marriage is regulated by laws and
ceremonies: therefore we believe, that all marriages in this church of Christ of
Latter Day Saints, should be solemnized in a public meeting, or feast, prepared for
that purpose: and that the solemnization should be performed by a presiding high
priest, high priest, bishop, elder, or priest, not even prohibiting those persons who
are desirous to get married, of being married by other authority.-We believe that it
is not right to prohibit members of this church from marrying out of the church, if
it be their determination so to do, but such persons will be considered weak in the
faith of our Lord and Savior Jesus Christ.

Marriage should be celebrated with prayer and thanksgiving; and at the solemnization,
the persons to be married, standing together, the man on the right, and the woman on
the left, shall be addressed, by the person officiating, as he shall be directed by
the holy Spirit; and if there be no legal objections, he shall say, calling each by
their names: ``You both mutually agree to be each other's companion, husband and wife,
observing the legal rights belonging to this condition; that is, keeping yourselves
wholly for each other, and from all others, during your lives." And when they have
answered ``Yes," he shall pronounce them ``husband and wife" in the name of the Lord
Jesus Christ, and by virtue of the laws of the country and authority vested in him:
``may God add his blessings and keep you to fulfil [fulfill] your covenants from
henceforth and forever. Amen."

The clerk of every church should keep a record of all marriages, solemnized in his 
branch.

All legal contracts of marriage made before a person is baptized into this church,
should be held sacred and fulfilled. Inasmuch as this church of Christ has been
reproached with the crime of fornication, and polygamy: we declare that we believe,
that one man should have one wife; and one woman, but one husband, except in case of
death, when either is at liberty to marry again. It is not right to persuade a woman
to be baptized contrary to the will of her husband, neither is it lawful to influence
her to leave her husband. All children are bound by law to obey their parents; and to
influence them to embrace any religious faith, or be baptized, or leave their parents
without their consent, is unlawful and unjust. We believe that husbands, parents and
masters who exercise control over their wives, children, and servants and prevent
them from embracing the truth, will have to answer for that sin.

We have given the above rule of marriage as the only one practiced in this church, to
show that Dr. J. C. Bennett's ``secret wife system" is a matter of his own
manufacture; and further to disabuse the public ear, and shew [show] that the said
Bennett and his misanthropic friend Origen Bachelor, are perpetrating a foul and
infamous slander upon an innocent people, and need but be known to be hated and
despise. In support of this position, we present the following certificates:-

We the undersigned members of the church of Jesus Christ of Latter-Day Saints and
residents of the city of Nauvoo, persons of families do hereby certify and declare
that we know of no other rule or system of marriage than the one published from the
Book of Doctrine and Covenants, and we give this certificate to show that Dr. J. C.
Bennett's ``secret wife system".

(page 939)

is a creature of his own make as we know of no such society in this place nor never 
did.

S. Bennett, N. K. Whitney,

George Miller, Albert Pettey,

Alpheus Cutler, Elias Higbee,

Reynolds Cahoon, John Taylor,

Wilson Law, E. Robinson,

W. Woodruff, Aaron Johnson.

We the undersigned members of the ladies' relief society, and married females do
certify and declare that we know of no system of marriage being practised [practiced]
in the church of Jesus Christ of Latter Day Saints save the one contained in the Book
of Doctrine and Covenants, and we give this certificate to the public to show that J.
C. Bennett's ``secret wife system" is a disclosure of his own make.

Emma Smith, President,

Elizabeth Ann Whitney, Counsellor (Counselor),

Sarah M. Cleveland, Counsellor (Counselor),

Eliza R. Snow, Secretary,

Mary C. Miller, Catharine Pettey,

Lois Cutler, Sarah Higbee,

Thirza Cahoon, Phebe Woodruff

Ann Hunter, Leonora Taylor,

Jane Law, Sarah Hillman,

Sophia R. Marks, Rosannah Marks,

Polly Z. Johnson, Angeline Robinson,

Abigail Works.

Missouri Law.-The Executive committee of the Am. A. S. Society have taken legal
advice in regard to what can de [be] done for Thompson, Work, and Burr, confined for
twelve years in the penitentiary of Missouri. The result is, that nothing can be done
for their relief-the case being quite out of the jurisdiction of the other courts.
The only thing which can possibly avail them is, for the governors of those States of
which they were citizens, to expostulate with the governor of Missouri, and obtain
some abridgement [abridgment] of the time. Whether they will do this is very
doubtful. This is a hard case; for it is admitted, even in Missouri, that they broke
no law except by a forced construction. Indeed, when the young men were arrested, it
was a long time before they could find any law under which to try them, and the law
they applied did not, and never was intended to have any relation to the
case.

We have copied the foregoing article for the purpose of showing that the State of
Missouri, is not governed by law in her disposition of those that are considered
offensive. If ``the young men broke no law," and the law by which they were tried had
no relevancy to the case, how could they be sent to the penitentiary for twelve
years, except upon mob law, or despotic assumption? It is well such cruel cases, as
too often occur in Missouri, begin to attract the attention of some more sensible
portions of the American public. The church of Latter-Day Saints will not be the only
people, who complain of injustice and oppression from the people and government of
Missouri. We care nothing about abolitionism, and have nothing to do with it, but we
do care about the honor and virtue of our country, and want an equal enjoyment of
rights and privileges from the banker to the beggar; from the president to the
peasant:-but when wicked men bear rule the people mourn.

We certainly take pleasure in presenting to our readers, the following well directed
hit on Miller's Sectarian Millennium. It appears in the Olive Branch of Boston, and
if the editors had been as wise in their calculations from a plentiful harvest for
the people's salvation, as in their exposition of the Millennium's commencing in
April, 1843, they would have given one hint upon the voice of famine: but to the
article; viz:-

GOD'S WAYS ARE EQUAL. In his controversy with the ancient Jews God said, ``My ways are
equal, your ways are unequal." On this declaration we have been led lately to
reflect, when looking over the country and seeing the immense harvest about to be
gathered in. Nature is yielding in an unusual manner, and the strong probability is
that two years' provisions are soon to be reaped from the earth. Why is this? We know
that the All-wise Giver of good things has in time past sent plentiful years, but
they were to supply the necessities of his creatures in years of scarcity which were
to follow. In this he showed his ways to be equal. It was so with the seven years of
plenty in Egypt, which were followed by seven years of famine. This was an equal
balance of year for year; and no doubt this balance has always been kept up, the
surplus of one year supplying the deficiency of another. Here all is equal. Now our
reflections on this subject led us to propose the following question for the
consideration and answer of those who believe that this is the last year of the
world's existence. If the present is

(page 940)

to be the last year of the world, and God should supply the inhabitants thereof with
a large amount of food beyond the power of consumption, the present year, where is
the evidence of his wisdom, or of the truth of that declaration-``My ways are
equal?"

Joe Smith was seen on the 3d inst., on his way to Galena, and it was thought he would 
push for Canada. His influence is on the wane most evidently.-St. Louis Picket Guard.

It is a great pity that humbuggery was not on the wane too. Joe Smith is at his 
residence in Nauvoo, attending to and administering the droppings of Mormon 
beneficence. Apropos-would it not be a more wise course for the press abroad 
to drop this nonsensical jargon about the Mormons-let them pursue their vocations 
after their own modes, customs and consciences, than to be eternally poking sharp 
sticks at a harmless inoffensive sect? Surely we should think so. What say you, 
friend Whitney?

(Times and Seasons Vol 3 No 23)