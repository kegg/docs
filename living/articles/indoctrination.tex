\chapter{Indoctrination}

I never thought I would end up writing something like this. I suppose it was
really inevitable. Truth claims come and go like a dime a dozen and everyday
there are more people seeking the truth and yet they find more questions.
Official authorities don't ever have any answers regarding these truth
questions. So where does that leave us? I suppose it leaves us waiting and
wanting something to believe in that is true. Something to be able to grab hold
of and say ``yes, this is it!"

Instead we are left to that which we cannot tell is truth. It feels like
something that isn't truth. Something so beyond the truth that we simply cannot
reason with it anymore. It is one to drive a person mad.

A little bit about me. I was born in the covenant. That is, my parents were
sealed in the temple when I was born. I grew up in the church  my entire life. I
served a mission, got married in the temple. Kept my nose clean for most of it.
Sure there were bumps in the road as people have. But nothing I didn't overcome
through the proper channels.

I first ran into what is known as ``anti" material while on my mission. An
investigator had some pamplets. We explained it was incorrect information and
tossed it in the trash without even looking at it. Hey, we were there doing the
Lord's work right? So yeah, it felt like the right thing to do. Toss it away
don't look back.

Oh how I would love to go back in time to that moment. I would have read through
it. Looked it over and saw what there was to see. But I was playing the good
role of missionary. There was no reason for me not to. Little did I know I would
end up with the knowledge that I have years later. What a fool I had been.

I have never felt comfortable with the church. I don't know if it's just because
I didn't enjoy going to church as a kid? I don't know. I do know that when I
began studying things out in my mind from the resources, as we're directed to
from the scriptures.\footnote{D\&C 9:8} I know there are issues with church 
history and some of the doctrine taught by the LDS faith.

There's no denying it. How can I deny the witness I have received regarding it?
I can't. I must move forward with my head high knowing that I am doing the right
thing with my life.

The interesting part about all of this is all the opposition to researching the
truth. People in church roll their eyes when they hear that people are having
issues with church history. If there wasn't anything that was so alarming
about church history, I suppose people wouldn't eyeroll. There's a reason for
all of this isn't there? A reason research is causing such a fuss? I would like
to think there is. If not? Then all of this research is being done in vain.

Praying for the truth has never been beneficial to me. I have taken Moroni's
challenge\footnote{Moroni 10:3-5} as it were, and nothing ever came of it. 
Did I simply lack faith because I didn't receive an answer? Did I not pray hard 
enough? What exactly was the reason the heavens fell silent as I offered up 
my prayer?

We are taught that in order to receive revelation from God we have to pray.
We have to be humble enough to allow God to answer our prayers. I don't know how
much praying I can do on the subject before I grow weary in prayer unto God. If
He is there and he is listening? I haven't heard a word from Him regarding any
of this.

Feeling that the heavens are closed off to you is not a comforting feeling at
all. It is a lonely feeling. A feeling that no one is out there listening to
your prayers and that they simply don't care. If that's the case? I'm not sure I
even want to be part of this church any longer. It feels like I've wasted my
time already.

How long must a person pray and continue to pray for truth before the answers
come? Feeling alone because of it all is not beneficial. God has promised He
will never leave us, and yet ``common" people such as myself have not been able
to communicate with the heavens as it were. I'm not asking for a sign, because
you're not supposed to ask for signs. But it would be nice to have a concrete
answer about all of this. Even if it is from church leaders. Yet all they say is
to continue having faith. Nothing is needed beyond that. It's a shame really
that they won't answer the questions people have. What harm is done by answering
a question or two?

It's interesting, people deem certain materials ``anti". I call them that 
because they call the materials that. In reality is truth ``anti"? Who's to 
claim what is ``anti" vs. what is actual truth?

Shall we get a definition of ``anti"? I think we shall. Google states:

\begin{displayquote}
an-ti

preposition

1. opposed to; against

adjective \textit{informal}

1. opposed

noun \textit{informal}

1. a person opposed to a particular policy, activity, or idea
\end{displayquote}

There are days, I admit, it would be nice to be able to simply go back and
unlearn all of this. To be able to forget about everything I ever read. Yet the
truth is out there and what has been read and seen, cannot be unseen or unread.
It's not an easy road. To say I take any of this lightly would be false.

So here we are. Simply trying to figure out the truth of all things as it were.
Ask questions when necessary, and hopefully have the ability to continue to move
forward no matter what obstacale gets in our way. Not that a lot of obstacles
are expected, yet here I am simply trying to find the truth.

If the truth is found? Then I will accept it without hesitation. If the truth is
not found and all of this is for naught? Then I shall chalk it up to a learning
experience and will rightfully shred the data and toss it into the trash. It's
not a difficult thought process.

Either all of it is true or none of it is true. There really can be no partials
when it comes to the Kingdom of God can there? If that were the case, then God
would not be perfect. But He is a perfect being. There can be no chaos or
confusion when it comes to the truth.\footnote{1 Corinthians 14:33 - For God is 
not the \textit{author} of confusion, but of peace, as in all churches of 
the saints.}

There is a term called ``controlling the narrative". Which means you tell the
story your way before someone else tells it. Sometimes if the other person gets
to telling the story, they can tell the story better. By ``controlling the
narrative",  you are able to keep things to a more intimiate level and keep
people coming back to you instead of other sources.\footnote{See usatoday.com, 
The importance of `controlling the narrative', Michael Wolff}

This is what the Church of Jesus Christ of Latter-day Saints attempts to do, but
they execute it quite poorly. Instead of being upfront and open with people,
they tell people not to worry and to have more faith, which pushes people away
to the point where they seek out other sources for the truth.

You can probably see where this might run into issues down the road for people.
For the longest time, the church boasts of a rich history. They claim to
have the truth, the fullness of the restored gospel of Jesus Christ, 
and the truth is meant for all to learn from. However, when critics of the 
church come forward with issues or questions regarding the truth the church 
backs into a corner and pulls out the claws. You will either support their 
narriative or you will be quiet about the subject. There is no room for debate.

\begin{flushright}
Kyle Eggleston

May 14, 2018

Thinking Through The Light
\end{flushright}