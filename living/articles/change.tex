\chapter{Things Change}

\begin{displayquote}
In the beginning was the word. The word was with God. The word was of 
God.\footnote{John 1:1} 
\end{displayquote}

Jesus Christ was the \textit{word} that is spoken of.

It could be said that Jesus was God. But only in the sense that he was the 
God of the Old Testament.\cite{otStudentManual} However, that's not how the
words were originally written:

\begin{displayquote}
In the beginning was the gospel preached through the Son. And the gospel was 
the word, and the word was with the Son, and the Son was with God, and the Son 
was of God.\footnote{JST John 1:1}
\end{displayquote}

So, why the change? We're told that many plain and precious truths of the bible 
had been removed and lost,\footnote{1 Nephi 13:26-27} from the \textit{``great 
and abominable church''}.

There was a council of Carthage (397) in which it was decided the books to
be considered canon for the Bible. The books were named and no books have
been added to the Bible since.

Who was this ``great and abominable church"?

That question is still up for debate. Obviously it's the church of the devil. 
But is there a church standing today that classifys as that? I'd rather not 
go into that. We know what Bruce R. McConkie thought about it in Mormon 
Doctrine, that theory had later changed, and was removed from the book 
completely. Mormon Doctrine is no longer in Deseret Bookstore shelves. I wonder 
why that is.\footnote{[Under the heading, ``Church of the Devil," Apostle Bruce 
R. McConkie lists:] "The Roman Catholic Church specifically—singled out, set 
apart, described, and designated as being ‘most abominable above all other 
churches’ (I Ne. 13:5)" (Mormon Doctrine, 1958, 129).}

With changing times come other changes that people make. I suppose this article 
is all about change isn't it.

People change as time changes. What was right back in the 1800s or the 1600s, 
isn't right now. Hanging a witch, for example, people got that from Exodus 
22:18.\footnote{Thou shalt not suffer a witch to live.} But, we learn from the 
Joseph Smith Translation of the Bible, that it's not the word 
witch.\footnote{Thou shalt not suffer a \textit{murderer} to live. 
[JST Exodus 22:18]} Quite different between a witch and a murderer right?

Changes happen because men polute the words of God.\footnote{Mormon 8:36,38}

An example of possible change is with the first ``Eve" who was known as
``Lilith". There isn't much beyond what has been said about her from different
sources. It is an interesting story for a possible explaination of two
creation accounts in the book of Genesis. (See Appendix A: Lilith)

So, why would God allow these changes? We are told He allows people to have
agency, yet one would think He wouldn't allow men to pollute His holy word?
I suppose it is neither here nor there. That is fine and well.

I should point out that not all changes are evil. Not all changes come from 
Satan, the Devil, the Father of all lies. There are some changes in life that 
are actually good. Some changes that come because change was needed. You can see 
it in history. If you don't know what kind of changes I'm talking about, 
seriously go crack open a history book and see all there is to see and learn 
about.

It should also be pointed out that I question at times. If the fullness of the 
gospel was restored, why does there need to be change? Why wasn't it that way 
from the beginning? Herein, I shall go over some changes which have occurred
over the course of history.

If things need changing, I believe they shouldn't have been in their original
form to begin with. But again, that is my thought process on the matter.