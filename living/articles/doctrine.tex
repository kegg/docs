\section{Doctrine of Christ}

We are told in the Book of Mormon that the Doctrine of Christ is fourfold:

\begin{enumerate}
  \item Faith
  \item Repentance
  \item Baptism
  \item Receive the Holy Ghost
\end{enumerate}

This is found in 2nd Nephi 31-32, there it states the Doctrine of Christ and the
results of receiving the Doctrine of Christ. Covenants with established blessings
attached to them.

The Book of Mormon then reiterates it in 3rd Nephi chapter 11. Again it establishes
Christ's Doctrine. We are to be baptized in Christ's name. But near the end of the
chapter, it says that ``whoso shall declare more or less than this, and establish it
for my doctrine, the same cometh of evil, and is not built upon my rock."\footnote{
3 Nephi 11:40
} It goes on to say how the person will be thrust down to hell etc.

The Church of Jesus Christ of Latter-day saints declares its doctrine to be more than
that which Christ has spoken, how does the church get around it? What are their
explainations for adding to the doctrine of Jesus Christ?

I find it troubling the churche's own scripture even says not to add to it, and yet
they have. I suppose one could argue the church isn't adding to the Doctrine of
Christ, but adding to the doctrine as a whole. The doctrine of Christ remains
untouched. But is not Christ the church and that doctrine would be the same?

To me, these scriptures teach that we are to repent of our sins, be baptized and
become like a little child. We are then able to inherit the kingdom of God. It
doesn't say anything about having to go to the temple, pay tithing etc. Those things
are not requirements to be with God. Yet the church claims those are requirements,
that one cannot dwell with God without having all of those necessary ordinances done.

Are not those other ordinances additions to the Doctrine of Christ? The same thing
which Jesus had warned against having additions to?

According to ``Teaching in the Savior's way" there are nine doctrinal principles the
church has. They are:

\begin{enumerate}
  \item Godhead
  \item Plan of Salvation
  \item Atonement of Jesus Christ
  \item Dispensation, Apostasy, and Restoration
  \item Prophets and Revelation
  \item Priesthood and Priesthood Keys
  \item Ordinances and Covenants
  \item Marriage and Family
  \item Commandments
\end{enumerate}

Is this not adding to Christ's Doctrine, that which is required to get into heaven?
If Christ's Doctrine is easy and simple to understand so everyone can take part and
if they try hard enough, they can come back to God. Which is what God want's right? I
can't see a God forcing His children to jump through hoops in order to get back to
His presence. Well, the church would tell you they aren't forcing anyone to jump
through hoops, if they don't want to play by the rules they'll be excommunicated.

If a man professes Christ to be their Lord and Savior, repents of their sins gets 
baptized and receives the Holy Ghost; why can't they be saved? Why does there need to
be additional doctrines and teachings out there?

Some will claim it goes back to authority. Who gave Adam authority to baptize? Why is
there no record of Adam baptizing Eve in the Bible? Wouldn't that be one of the
truths transcribers would leave in there if it's so important?

Oh, according to the Joseph Smith Translatin of the bible, chapter 6 of Genesis
states:

\begin{displayquote}
64 And it came to pass, when the Lord had spoken with Adam, our father, that Adam 
cried unto the Lord, and he was caught away by the Spirit of the Lord, and was 
carried down into the water, and was laid under the water, and was brought forth 
out of the water.

65 And thus he was baptized, and the Spirit of God descended upon him, and thus he 
was born of the Spirit, and became quickened in the inner man.

66 And he heard a voice out of heaven, saying: Thou art baptized with fire, and with 
the Holy Ghost. This is the record of the Father, and the Son, 
from henceforth and forever;\footnote{JST Genesis 6:64-66}
\end{displayquote}

Well that brings about an interesting question. The Holy Ghost baptized Adam, and
then descended upon him? Without someone laying hands on Adam to give him the gift of
the Holy Ghost? So it's not even the same way the LDS Church teaches it to be? Surely
God would have set the correct course for baptism with Adam, wouldn't He have?

There is of course the conflicting account where Alma baptizes himself:

\begin{displayquote}
12 And now it came to pass that Alma took Helam, he being one of the first, and went 
and stood forth in the water, and cried, saying: O Lord, pour out thy Spirit upon thy 
servant, that he may do this work with holiness of heart.

13 And when he had said these words, the Spirit of the Lord was upon him, and he 
said: Helam, I baptize thee, having authority from the Almighty God, as a testimony 
that ye have entered into a covenant to serve him until you are dead as to the mortal 
body; and may the Spirit of the Lord be poured out upon you; and may he grant unto 
you eternal life, through the redemption of Christ, whom he has prepared from the 
foundation of the world.

14 And after Alma had said these words, both Alma and Helam were buried in the 
water; and they arose and came forth out of the water rejoicing, being filled with 
the Spirit.

15 And again, Alma took another, and went forth a second time into the water, and 
baptized him according to the first, only he did not bury himself again in the 
water.\footnote{Mosiah 18:12-15}
\end{displayquote}

Why didn't the Holy Ghost baptize Alma like he did Adam? Where did Alma get the
authority to baptize \textit{before} being baptized himself? If all he did was call
upon God and asked for the authority and God gave the authority, why can't that be
done today? Why must it be given by someone who has authority by the laying on of
hands?

Even when Joseph Smith and Oliver Cowdry were baptized, the Holy Ghost seemed to have
descended upon them:

\begin{displayquote}
``We experienced great and glorious blessings from our Heavenly Father. No sooner had 
I baptized Oliver Cowdery, than the Holy Ghost fell upon him, and he stood up and 
prophesied many things which should shortly come to pass. And again, so soon as 
I had been baptized by him, I also had the spirit of prophecy, when, standing up, 
I prophesied concerning the rise of this Church, and many other things connected 
with the Church, and this generation of the children of men. We were filled with 
the Holy Ghost, and rejoiced in the God of our salvation"\footnote{
Joseph Smith—History 1:73}
\end{displayquote}

Granted, some would say that was temporary and they would receive the gift of the
Holy Ghost at a later time. But interesting, after they were baptized the Holy Ghost
descended upon them, like it did when Jesus got baptized.

Again, why did the Holy Ghost come to people in this manner? Why didn't God send
angels down to baptize Joseph and Oliver. The scriptures state Joeph and Oliver
received the authority to baptize before they baptized each other, after which they
conferred the priesthood in the normal way. The reason given? Because they were
commanded to.\footnote{Excuse me, but what? John the Baptist gave them the priesthood
before they baptized each other? That is not the way it's taught in the church. Being
told it was given by way of commandment? Honestly? Brings back the question, what
makes them different in getting the priesthood than someone else (like Alma) who prayed
to God and asked for the ability to do it, being moved upon by the Holy Ghost.}

It would appear the doctrine could be in any form or order as long as God orders it.
But does not God even follow rules of order as far as the priesthood is concerned?
Let's break down the order by which Joseph Smith was ordained etc.

\begin{enumerate}
  \item John conferrs the priesthood upon Joseph and Oliver
  \item Joseph and Oliver baptize each other
  \item John tells Joseph and Oliver to lay hands on each other to re-confer the
  priesthood which he had bestowed upon them.
\end{enumerate}

Here's a quote from Joseph Fielding Smith regarding the event:

\begin{displayquote}
(John the Baptist) after conferring the Priesthood, instructed Joseph and Oliver to 
go down into the water and baptize each other. After which they were to lay hands 
upon each other and re-confer the Priesthood which he had bestowed upon them. There 
are two reasons why they should be commanded to do this thing. First, to confer the 
Priesthood before baptism, is contrary to the order of the organized Church, 
therefore they were commanded to confer the Priesthood upon each other in the regular 
way; after they were baptized. Second, the angel did for them that which they could 
not do for themselves. There was no one living in mortality who held the keys of this 
Priesthood, therefore it was necessary that this messenger, who held the keys of the 
Aaronic Priesthood in the Dispensation of the Meridian of Time, should be sent to 
confer this power. It is contrary to the order of heaven for those who have passed 
beyond the veil to officiate and labor for the living on the earth, only wherein 
mortal man cannot act, and thereby it becomes necessary for those who have passed 
through the resurrection to act for them. Otherwise John would have followed the 
regular order, which is practiced in the Church, and would have first baptized 
Joseph Smith and Oliver Cowdery and then conferred upon them the Aaronic 
Priesthood.\cite{essentials}
\end{displayquote}

According to that quote, John wasn't a resurrected being and couldn't perform the
baptism. But how did he lay hands on their heads if he had no body to do so? Why
could he perform that rite if he couldn't even baptize them?