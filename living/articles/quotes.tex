\chapter{Quotes}

\begin{displayquote}
Say nothing of my religion. It is known to my God and myself alone. Its evidence 
before the world is to be sought in my life; if that has been honest and dutiful 
to society, the religion which has regulated it cannot be a bad one.\footnote{
Thomas Jefferson
}
\end{displayquote}

Quotes abound in the world of religion. There are so many religious quotes out 
there, some days I do not know where some begin and others end. They are that 
which bring forth the breath of life into people. Allowing them to try and see 
all that is out there.

Granted, not all quotes are faith promoting. There are quotes out there which 
people would rather be buried never to see the light of day. But still those 
quotes exist for a reason. Sometimes those reasons are more than we can grasp 
at the current moment in time.

\begin{displayquote}
If we have the truth, it cannot be harmed by investigation. 
If we have not the truth, it ought to be harmed.\footnote{
President J. Reuben Clark}
\end{displayquote}

The truth is out there. One just has to want to look for it. It isn't hiding. 
In most instances the truth is there, in plain sight. If we do not actually 
look for the truth or seek it out, we cannot find the truth. It will seldom 
come to find us.

\begin{displayquote}
The man who cannot listen to an argument which opposes his views either has a 
weak position or is a weak defender of it. No opinion that cannot stand 
discussion or criticism is worth holding. And it has been wisely said that 
the man who knows only half of any question is worse off than the man who 
knows nothing of it. He is not only one sided, but his partisanship soon 
turns him into an intolerant and a fanatic. In general it is true that nothing 
which cannot stand up under discussion and criticism is worth
defending.\footnote{James E. Talmage}
\end{displayquote}

\begin{displayquote}
We desire that the brethren and sisters will all feel the responsibility of 
expressing their feelings in relation to the propositions that may be put 
before you. We do not want any man or woman who is a member of the Church to 
violate their conscience. We would like all to vote as they feel, whether for 
or against.\footnote{President Joseph F. Smith}
\end{displayquote}

Men and Women are free to choose according to their own thoughts. That is the 
will and mind of the Lord. For we were all given agency th at we might choose 
good from evil. Without having such a choice, we wouldn't learn or grow from 
that which comes across our path in this life.

There are other quotes by church leaders which are not quite in tune with that 
which we have today. Some quotes have been disavowed, others have been simply 
said the person speaking was saying such things as a man.

\begin{displayquote}
Live a good life. If there are gods and they are just, then they will not care 
how devout you have been, but will welcome you based on the virtues you have 
lived by. If there are gods, but unjust, then you should not want to worship 
them. If there are no gods, then you will be gone, but will have lived a noble 
life that will live on in the memories of your loved ones.\footnote{
Marcus Aurelius}
\end{displayquote}

Isn't that what this life is all about? Living a good life? Shouldn't that be
what this life is all about? Do unto others as you would have them do unto you?
Come on now, it can't be this difficult.

\begin{displayquote}
Cherish your doubts, for doubt is the attendant of truth. Doubt is the key to the
door of knowledge; it is the servant of discovery. A belief which may not be
questioned binds us to error, for there is incompleteness and imperfection in every
belief. Doubt is the touchstone of truth; it is an acid which eats away the false.
Let no one fear the truth, that doubt may consume it; for doubt is a testing of
belief. The truth stands boldly and unafraid; it is not shaken by the testing:
For truth, if it be truth, arises from each testing stronger, more secure. Those
that would silence doubt are filled with fear; their houses are built on shifting
sands. But those who fear not doubt, and know its use, are founded on rock.
They shall walk in the light of growing knowledge; the work of their hands shall
endure. Therefore let us not fear doubt, but let us rejoice in its help: It is to be
the wise as a staff to the blind; doubt is the attendant of truth.\footnote{
Robert T. Weston}
\end{displayquote}

The LDS (Mormon) Church has an issue of anti-intelectualism. See the following:

\begin{displayquote}
He [Satan] wins a great victory when he can get members of the church to speak
against their leaders and to do their own thinking. When our leaders speak, the
thinking has been done. When they propose a plan – it is God's Plan. When they point
the way, there is no other which is safe. When they give directions, it should mark
the end of controversy, God works in no other way. To think otherwise, without 
immediate repentance, may cost one his faith, may destroy his testimony, and leave 
him a stranger to the Kingdom of God.\footnote{
Ward Teacher's Message, Deseret News, Church Section, p. 5, May 26, 1945; see also 
Improvement Era, June 1945
}
\end{displayquote}

\begin{displayquote}
Always keep your eye on the President of the church, and if he ever tells you to do
anything, even if it is wrong, and you do it, the lord will bless you for it, but you
don't need to worry. The lord will never let his mouthpiece lead the people
astray.\footnote{Apostle Marion G. Romney, Conference Report, Oct. 1960, p. 78}
\end{displayquote}

\begin{displayquote}
When Elder Packer interviewed me as a prospective member of Brigham Young
University's faculty in 1976, he explained: ``I have a hard time with historians
because they idolize the truth. The truth is not uplifting; it destroys. I could tell
most of the secretaries in the church office building because that they are ugly and
fat. That would be the truth, but it would hurt and destroy them. Historians should
tell only that part of the truth that is inspiring and uplifting."\footnote{
D. Michael Quinn, “On Being a Mormon Historian (and Its Aftermath),” in George D. 
Smith, ed., Faithful History: Essays on Writing Mormon History, 1992, p. 76
}
\end{displayquote}

\begin{displayquote}
When the Prophet speaks the debate is over.\footnote{Apostle N. Eldon Tanner, 
Ensign, Aug. 1979, pp. 2-3.}
\end{displayquote}

\begin{displayquote}
Avoid those who would tear down your faith. Faith-killers are to be shunned. The
seeds which they plant in the minds and hearts of men grow like cancer and eat
away the Spirit.\footnote{Carlos E. Asay, October 1981 General Conference}
\end{displayquote}

\begin{displayquote}
No true Latter-day Saint will ever take a stand that is in opposition to what the 
Lord has revealed to those who direct the affairs of his earthly kingdom. No 
Latter-day Saint who is true and faithful in all things will ever pursue a course, 
or espouse a cause, or publish an article or book that weakens or destroys 
faith.\footnote{Apostle Bruce R. McConkie, October 1984 General Conference}
\end{displayquote}

\begin{displayquote}
Some things that are true are not edifying or appropriate to communicate. Readers 
of history and biography should ponder that moral reality as part of their effort to 
understand the significance of what they read.\footnote{Apostle Dallin H. Oaks, 
``Reading Church History," Ninth Annual Church Educational System Religious 
Educators' Symposium, August 16, 1985, Brigham Young University}
\end{displayquote}

\begin{displayquote}
There are three areas where members of the Church, influenced by social and political
unrest, are being caught up and led away. I chose these three because they have made
major invasions into the membership of the Church. In each, the temptation is for us
to turn about and face the wrong way, and it is hard to resist, for doing it seems
reasonable and right.

The dangers I speak of come from the gay-lesbian movement, the feminist movement
(both of which are relatively new), and the ever-present challenge from the so-called
scholars or intellectuals. Our local leaders must deal with all three of them with
ever increasingly frequency. In each case, the members who are hurting have the
conviction that the Church somehow is doing something wrong to members or that the
Church is not doing enough for them.\footnote{Apostle Boyd K. Packer, 
``Talk to the All-Church Coordinating Council," May 18, 1993}
\end{displayquote}

\begin{displayquote}
The Church warns its members against symposia and similar gatherings that include
presentations that (1) disparage, ridicule, make light of, or are otherwise
inappropriate in their treatment of sacred matters or (2) could injure the Church,
detract from its mission, or jeopardize its members’ well-being. Members should not
allow their position or standing in the Church to be used to promote or imply
endorsement of such gatherings.\footnote{Section 17.1.46 of Handbook 1 for Stake 
Presidents and Bishops.}
\end{displayquote}

\begin{displayquote}
Doubt your doubts before you doubt your faith.\footnote{Dieter F. Uchtdorf, 
October 2013 General Conference}
\end{displayquote}

\begin{displayquote}
Another claim we sometimes hear is that the leaders won’t answer our doubts. Doubts.
Here we need to define the difference between doubts and questions. Questions, when
asked with a sincere desire to increase ones understanding and faith, are again
encouraged. Such questions, questions we call them, are asked with the real intent of
better understanding and more fully obeying the will of the Lord. Questions are very
different from doubts. ... One difference between questions asked in faith and doubts
is that questions lead to faith and to revelation whereas doubts lead to
disobedience, which in turn renders people less able to receive revelation, or in
other words, doubt is darkness. Questions asked in faith lead to light.\footnote{
Apostle Dallin H. Oaks and Assistant Church Historian Richard E. Turley Jr. 
during the Boise Rescue on June 13th, 2015.
}
\end{displayquote}

\begin{displayquote}
Some who use personal reasoning or wisdom to resist prophetic direction give
themselves a label borrowed from elected bodies -- ``the loyal opposition." However
appropriate for a democracy, there is no warrant for this concept in the government
of God’s kingdom, where questions are honored but opposition is not.\footnote{
Apostle Dallin H. Oaks, April 2016 General Conference
}
\end{displayquote}

The church doesn't like people researching its history. Why is that exactly? What
does it have to hide that's so bad, it's afraid of letting people know its full
history?