\section{Appendix B: J. Reuben Clark: The Church Years}

By 1917, however, Reuben was asking himself some religious questions that took 
him years to resolve. In one personal memo he began, ``If we have truth, [it] 
cannot be harmed by investigation. If we have not truth, it ought to be harmed." 
From that premise he added the observation that scientists and lawyers 
(like himself) were not blindly believing and that they must refuse to be 
deceived by others or by their own wishful thinking. ``A lawyer must get at 
facts, he must consider motives -- he must tear off the mask and lay bare the 
countenance, however hideous. The frightful skeleton of truth must always be 
exposed ... [the lawyer] must make every conclusion pass the fiery ordeal of 
pitiless reason. If their conclusions cannot stand this test, they are false." 
During the same year the increasingly introspective lawyer asked himself the 
questions: Are we not only entitled, but expected to think for ourselves? 
Otherwise where does our free agency come in? His answer was a resounding: 
``If we are blindly to follow some one else we are not free agents.... 
That we may as a Church determine for ourselves our course of action, is 
shown by the Manifesto [abandoning the practice of polygamy]. We may not 
probably take an affirmative stand, i.e., adopt something new but we may 
dispense with something." Perhaps he had never before questioned the assumptions 
that lay behind some of the simple faith of his youth, but at midlife J. 
Reuben Clark, Jr. proclaimed that there must be no forbidden questions in 
Mormonism.

The directions to which his philosophy of religious inquiry led him were 
indicated in his musings about two essentials of Mormonism: the revelations of 
Joseph Smith, Jr. and the Church belief in progression toward godhood. As he 
examined the revelations in the Doctrine and Covenants concerning the structure 
of the Church government, Reuben Clark wondered to what extent Joseph Smith's 
reading or experience, ``his own consciousness," had contributed to what he set 
down, and when Reuben pondered the Mormon belief in the potential of individuals 
to attain the godly stature of their Father in Heaven, his logical mind boggled 
a bit. ``Is Space or occupied portions of it divided among various deities -- 
have they great `spheres of influence'? War of Gods -- think of wreck of matter 
involved -- if matter used -- or would it be a war of forces?" In his 
mid-forties, he regarded these as legitimate doctrinal inquiries but soon 
realized that each question concerning doctrine led to other questions, each 
of which was further removed from rational verification. Reuben soon came to the 
conclusion he described in later years to the non-Mormon president of George 
Washington University: ``For my own part I early came to recognize that for me 
personally I must either quit rationalizing ... or I must follow the line of 
my own thinking which would lead me I know not where."

But J. Reuben Clark soon recognized where an uncompromising commitment to 
rational theology would lead him, and he shrank from the abyss. 
``I came early to appreciate that I could not rationalize a religion for myself, 
and that to attempt to do so would destroy my faith in God," 
he later wrote to his non-Mormon friend. ``I have always rather worshipped 
facts," he continues,``and while I thought and read for a while, many of the 
incidents of life, experiences and circumstances led, unaided by the spirit of 
faith, to the position of the atheist, yet the faith of my fathers led me to 
abandon all that and to refrain from following it.... For me there seemed to be 
no alternative. I could only build up a doubt. --If I were to attempt to 
rationalize about my life here, and the life too come, I would be drowned in 
a sea of doubt."

All the confidence of J. Reuben Clark's commitment to rational inquiry in 
religious matters evaporated. He had once believed that in intellectual faith 
``we may not probably take an affirmative stand, i.e., adopt something new but 
we may dispense with something," but Reuben found that such an attempt could 
only lead to dispensing with everyting [sic]. As he cast about for some way of 
explaining his position to others, he discovered an anecdote about Abraham 
Lincoln, who justified reading the Bible despite his reputed agnosticism with 
the comment: ``I have learned to read the Bible. I believe all I can and take 
the rest on faith." To a friend, Reuben related the Lincoln story and added, 
``Substituting in the substance the words `our Mormon Scriptures,' you will 
have about my situation." He later commended that anecdote to a general 
conference of the Church. Convinced that no religious faith could withstand 
uncompromising intellectual inquiry, Reuben concluded that in Babylon as well 
as in Zion, the refusal to rationalize one's religious beliefs was the highest 
manifestation of faith.

\cite{clark}