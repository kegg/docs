\chapter{Protect the Children}

Jesus taught the people to not offend His little ones, and encouraged them all to
come unto Him.

There are those who would practice unrighteous dominion towards the children of the
church, and this is wrong. The children of the church must be protected against
predators within the church. It is sad that such a thought even needs to be brought
to people's attention, yet it is real.

Said of children:

\begin{displayquote}
``In some minds there seems to be an idea that there should be a different form of
blessing for children born of non-members and for those who are identified with the
Church; and it is from such sources that in the case of children belonging to members
of the Church `the blessings of Abraham, Isaac and Jacob' and all the attendant
favors are frequently conferred upon the child. This is all wrong. If we take the
example of our Lord and Redeemer, who is our pattern and whose example we cannot too
closely follow, we find that He blessed all who were brought to Him. We have no hint
that He asked whose children they were, or the standing or faith of their parents.
His remark was, `Suffer little children, and forbid them not, to come unto me, for of
such is the Kingdom of Heaven;' and He laid His hands upon them and blessed them. All
little children, no matter what their parentage may be, are innocent in the sight of
heaven, and they should be received as such and blessed as such." -George Q Cannon,
Latter-day Saints'.\footnote{Millennial Star 61 (March 30, 1899) and Juvenile
Instructor}
\end{displayquote}

There is also a policy\footnote{
This was later clarified as being revelation by Russel M. Nelson in a worldwide
devotional he gave. Becoming True Millennials, January 10, 2016, LDS.org
} put out by the church in their Handbook 1 for Stake Presidents, Bishops, Mission 
Presidents etc that will not allow a child who has same sex parents be baptized into 
the church of Christ. They cannot receive a name and a blessing 
either.\footnote{Handbook 1, 16.13}

Said of the policy change:

\begin{displayquote}
The First Presidency and Quorum of the Twelve Apostles counsel together and share all 
the Lord has directed us to understand and to feel individually and collectively. 
And then we watch the Lord move upon the President of the Church to proclaim the 
Lord's will.

This prophetic process was followed in 2012 with the change in minimum age for 
missionaries and again with the recent additions to the Church’s handbook, consequent 
to the legalization of same-sex marriage in some countries. Filled with compassion 
for all, and especially for the children, we wrestled at length to understand the 
Lord's will in this matter. Ever mindful of God’s plan of salvation and of His hope 
for eternal life for each of His children, we considered countless permutations and 
combinations of possible scenarios that could arise. We met repeatedly in the temple 
in fasting and prayer and sought further direction and inspiration. And then, when 
the Lord inspired His prophet, President Thomas S. Monson, to declare the mind of 
the Lord and the will of the Lord, each of us during that sacred moment felt a 
spiritual confirmation. It was our privilege as Apostles to sustain what had been 
revealed to President Monson. Revelation from the Lord to His servants is a sacred 
process, and so is your privilege of receiving personal revelation.\footnote{
Becoming True Millennials, January 10, 2016, LDS.org
}
\end{displayquote}

When the child turns eighteen years old, and moves out of their parents house, then
they can be baptized. In effect, they must turn away from their parents and disown
them.

This appears to go against the second Article of Faith which states:

\begin{displayquote}
We believe that men will be punished for their own sins, and not for Adam's 
transgression.\footnote{Articles of Faith 1:2}
\end{displayquote}

Then there is the story in the Bible:

\begin{displayquote}
And as Jesus passed by, he saw a man which was blind from his birth.

And his disciples asked him, saying, Master, who did sin, this man, or his parents, 
that he was born blind?

Jesus answered, Neither hath this man sinned, nor his parents: but that the works of 
God should be made manifest in him.

I must work the works of him that sent me, while it is day: the night cometh, 
when no man can work.

As long as I am in the world, I am the light of the world.

When he had thus spoken, he spat on the ground, and made clay of the spittle, 
and he anointed the eyes of the blind man with the clay,

And said unto him, Go, wash in the pool of Siloam, (which is by interpretation, 
  Sent.) He went his way therefore, and washed, and came seeing.\footnote{John 9:1-7}
\end{displayquote}

The man was not blind because of his parents actions, he was born blind so God might
show His work to the people. But the point is, because of his parents actions, he
wasn't cursed. His parents played no role in it, neither did he for he was innocent
at birth. That is how our children are in today's world. They are innocent. Protect
their innocence.

This topic alone could cover an entire book, and it has by some accounts. You can
read more about all of this by going to http://protectldschildren.org.