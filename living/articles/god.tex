\chapter{As God now is, man may be}

As was taught from teachings of a certain president of the Church of Jesus
Christ of Latter-day Saints, Lorenzo Snow taught:

\begin{displayquote}
As man now is, God once was:

As God now is, man may be.\footnote{In Eliza R. Snow Smith, Biography and 
Family Record of Lorenzo Snow (1884), 46; see also ``The Grand Destiny of Man," 
Deseret Evening News, July 20, 1901, 22.}
\end{displayquote}

This appears to be another thing has has gone under some change? For according
to the Church of Jesus Christ of Latter-day Saint's official news 
page,\footnote{http://mormonnewsroom.com} we don't follow that teaching anymore.

Let's pull a quote directly from a FAQ on that site:

\begin{displayquote}
\textbf{Do Latter-day Saints believe they can become ``gods"?}

Latter-day Saints believe that God wants us to become like Him. But this 
teaching is often misrepresented by those who caricature the faith. 
The Latter-day Saint belief is no different than the biblical teaching, 
which states, ``The Spirit itself beareth witness with our spirit, 
that we are the children of God: and if children, then heirs; heirs of God, 
and joint-heirs with Christ; if so be that we suffer with him, that we may be 
also glorified together" (Romans 8:16-17). Through following Christ's 
teachings, Latter-day Saints believe all people can become ``partakers of the 
divine nature"
(2 Peter 1:4).\footnote{https://www.mormonnewsroom.org/article/mormonism-101}
\end{displayquote}

If church members are not taught that we can become Gods, what was the 
revelation in Doctrine and Covenants 76 for? It teaches of the three kingdoms
of God, specifically the Celestial, Terrestrial, and Telestial kingdoms.

There's a scripture in that, verse 58 that states:

\begin{displayquote}
Wherefore, as it is written, they are gods, even the sons of 
God—\footnote{D\&C 76:58}
\end{displayquote}

Then there's the scripture in section 132:

\begin{displayquote}
And again, verily I say unto you, if a man marry a wife by my 
word, which is my law, and by the new and everlasting covenant, 
and it is sealed unto them by the Holy Spirit of promise, by him 
who is anointed, unto whom I have appointed this power and the keys
of this priesthood; and it shall be said unto them—Ye shall come 
forth in the first resurrection; and if it be after the first 
resurrection, in the next resurrection; and shall inherit thrones, 
kingdoms, principalities, and powers, dominions, all heights and 
depths—then shall it be written in the Lamb’s Book of Life, that 
he shall commit no murder whereby to shed innocent blood, and if 
ye abide in my covenant, and commit no murder whereby to shed innocent 
blood, it shall be done unto them in all things whatsoever my servant 
hath put upon them, in time, and through all eternity; and shall be of 
full force when they are out of the world; and they shall pass by the 
angels, and the gods, which are set there, to their exaltation and 
glory in all things, as hath been sealed upon their heads, which 
glory shall be a fulness and a continuation of the seeds 
forever and ever.

Then shall they be gods, because they have no end; therefore shall 
they be from everlasting to everlasting, because they continue; then 
shall they be above all, because all things are subject unto them. 
Then shall they be gods, because they have all power, and the 
angels are subject unto them.\footnote{D\&C 132:19-20}
\end{displayquote}

This is describing those who belong to the Celestial Kingdom. If we are not to
become Gods, as is stated in the Mormon Newsroom article, then what is it? Which
source does one believe pertaining to their eternal salvation, given that they
``come forth in the resurrection of the just."\footnote{D\&C 76:50} 

Past prophets speaking vs current policy teaching. Which is true and which is
false? Again, why a change? Why can't the church stand boldly in what they have
taught to be the truth and continue with it? Why must changes need to be made?

If God is the same yesterday, today, and forever why does He change? Is it
simply because times change? It is taught that God must follow the laws of
science and the other material laws when it comes to creation etc., yet if He
changes things now, or allows men to change things, how are we supposed to know
He won't change things after we have died?

\begin{displayquote}
For do we not read that God is the same yesterday, today, and forever, and 
in him there is no variableness neither shadow of 
changing?

And now, if ye have imagined up unto yourselves a god who doth vary, and in 
whom there is shadow of changing, then have ye imagined up unto yourselves a 
god who is not a God of miracles.\footnote{Book of Mormon 9:9}
\end{displayquote}

So, which is it? Is God a God of mircales? Or is He changing as the times here
on earth see fit?

In the book Gospel Principles, in a chapter on Exaltation, it once said:

\begin{displayquote}
\textbf{WHAT IS EXALTATION?}

Exaltation is eternal life, the kind of life that God lives. He lives in great
glory. He is perfect. He possesses all knowledge and all wisdom. He is the
father of spirit children. He is a creator. We can become Gods like our Heavnly
Father. This is exaltation.

If we prove faithful and obedient to all the commandments of the Lord, we will
live in the highest degree of the celestial kingdom of heaven. We will become
exalted, just like our Heavenly Father. Exaltation is the highest reward that
our Heavenly Father can give his children. The Lord has said that exaltation
is the greatest gift of all the gifts of 
God (see D\&C 14:7).\cite[pp. 289-290]{gp}
\end{displayquote}

That text was from a 1979 revised edition of the book, originally recommended
to missionaries as part of the Missionary Reference Library. I carred it on 
my mission and have access to the book. When compared to a later version, 
the narriative has changed. I will put an elipses in to show where the 
change is:

\begin{displayquote}
\textbf{What is exaltation?}

Exaltation is eternal life, the kind of life God lives. He lives in great glory. 
He is perfect. He possesses all knowledge and all wisdom. He is the 
Father of spirit children. He is a creator. We can become [...] like our 
Heavenly Father. This is exaltation.

If we prove faithful to the Lord, we will live in the highest degree of the 
celestial kingdom of heaven. We will become exalted, to live with our 
Heavenly Father in eternal families. Exaltation is the greatest gift that 
Heavenly Father can give His children (see D\&C 14:7).\cite[275-280]{gp2}
\end{displayquote}

You'll notice they took out the word Gods in that first paragraph. It has 
changed from telling us that we can become Gods to just that we can become
like our Heavenly Father. No promise of Godhood there.

The second paragraph, well you can see the change for yourself. I believe it 
speaks for itself quite well.

So, what brings about such changes? They were fine for earlier members of the
church. Why would they be changed now? It should be considered a doctrinal 
change. The emphasis has been changed over the years to show living with God 
in the post-mortal life, instead of becoming Gods ourselves.

I find a lot of the older doctrine as it were isn't taught much in these much 
later days. I wonder why that is. Are they too being tossed aside as people 
speaking as a man? I would doubt so. It is interesting that no one has spoken 
much in General Conference of the King Follett sermon lately.

\begin{displayquote}
God himself was once as we are now, and is an exalted man, and sits enthroned 
in yonder heavens! That is the great secret. If the veil were rent today, 
and the great God who holds this world in its orbit, and who upholds all 
worlds and all things by His power, was to make himself visible—I say, if 
you were to see him today, you would see him like a man in form—like yourselves 
in all the person, image, and very form as a man; for Adam was created in the 
very fashion, image and likeness of God, and received instruction from, and 
walked, talked and conversed with Him, as one man talks and communes with 
another.\footnote{King Follett Sermon, Joseph Smith Jr.}
\end{displayquote}

In it we are taught that God was once a man, which is the first part of Snow's 
couplet. Yet this is not openly widely taught these days. So yet another change 
has easily taken place. This is not to say it is not known, for the text is out 
there to be found. But it is not actively taught.

If God was once man that means he has a Father in heaven who is his God right? How
far back does that go? For eternity? If what is taught is true, then yes for
eternity. There was never a beginning and there will never be an end.

Well, what about those quotes where it says there are no other Gods than God? If 
there are no other Gods, and there is only one God, how can God have a God etc? 
How can God have been a man then?

There was a set of lectures, doctrine, included in the 1835 Doctrine and Covenants.
Hence where the doctrine comes from the D\&C. In the fifth lecture, it talks about
the Godhead.

There are a few interesting details in this lecture. They are the following:

\begin{enumerate}
  \item God is a spirit
  \item There are two members of the Godhead
\end{enumerate}

1. God is a spirit

\begin{displayquote}
There are two personages who constitute the great, matchless, governing and supreme
power over all things...They are the Father and the Son: The Father being a personage
of spirit, glory and power: possessing all perfection and fulness: The Son, who was
in the bosom of the Father, a personage of tabernacle, made, or fashioned like unto
man, or being in the form and likeness of man, or, rather, man was formed after his
likeness, and in his image;\footnote{Lecture Fifth, Lectures on Faith}
\end{displayquote}

Did you catch that? It says plain as day that God is a ``personage of spirit". Yet in
the Doctrine and Covenants, we are taught that God has a body of flesh and bone. So,
which is it? Another conflict of teachings by Joseph Smith.

At the end of the lecture, there is a question and answer section. This is where we
come across the other issue.

2. There are two members of the Godhead

\begin{displayquote}
Question 3: How many personages are there in the Godhead?

Two: the Father and the Son. (5:1)

Question 4: How do you prove that there are two personages in the Godhead?

By the Scriptures. Genesis 1:26: (Also 2:6): And the Lord God said unto the Only
Begotten, who was with him from the beginning, Let us make man in our image, after
our likeness: — and it was done. Genesis 3:22: And the Lord God said unto the Only
Begotten, Behold, the man is become as one of us: to know good and evil. John 17:5:
And now, O Father, glorify thou me with thine own self with the glory which I had 
with thee before the world was. (5:12)\footnote{Lecture Fifth, Lectures on Faith}
\end{displayquote}

So, Joseph Smith taught there are only two members of the Godhead.

It conflicts what has been taught by later LDS teachings that there are three members
of the Godhead, namely God, Jesus, and the Holy Ghost.

Why the change? Did Joseph Smith know what he was teaching? He saw God right? Why did
he teach that God was a spirit then?