\chapter{Book of Abraham Translation}

The opening of the Book of Abraham, I suppose you could call it the introduction 
(how small it is), says the following:

\begin{displayquote}
A Translation of some ancient Records that have fallen into our hands from the
catacombs of Egypt. The writings of Abraham while he was in Egypt, called the Book of
Abraham, written by his own hand, upon papyrus.\footnote{Book of Abraham Introduction}
\end{displayquote}

From that introductoin alone, it indicates that the book was translated from
Abraham's writings on papyrus. It was written by his own hand. That's what I was
taught growing up. Sounds right.

Then the Church of Jesus Christ of Latter-day Saints comes out with an essay titled 
``Translation and Historicity of the Book of Abraham."

There's a paragraph in it that pulls into question the validity of what the
introduction claims:

\begin{displayquote}
None of the characters on the papyrus fragments mentioned Abraham's name or any of 
the events recorded in the book of Abraham. Mormon and non-Mormon Egyptologists 
agree that the characters on the fragments do not match the translation given in 
the book of Abraham, though there is not unanimity, even among non-Mormon scholars, 
about the proper interpretation of the vignettes on these fragments. Scholars have 
identified the papyrus fragments as parts of standard funerary texts that were 
deposited with mummified bodies. These fragments date to between the third 
century B.C.E. and the first century C.E., long after Abraham lived.
\footnote{Translation and Historicity of the Book of Abraham, Gospel Topic Essay,
LDS.org}
\end{displayquote}

So not only were the fragments dated long after Abraham lived, they don't even match
what the translation even said? Further inspection indicates that the fascimilies
don't match up with anything either. They were typical funeral documents.

According to the church, we have a book that was written on papyrus by a man who
couldn't have written the book on papyrus because it was written long after  Abraham
lived. Just trying to wrap my head around that thought. It doesn't sit will with me.

\section{Translation Doesn't Mean Translation}

According to the Joseph Smith Papers website, translation doesn't actually mean
translation.

\begin{displayquote}
As used in this series, translation does not refer to conventional translations, 
such as Smith’s exercises in the study of Hebrew.\footnote{
https://www.josephsmithpapers.org/intro/revelations-and-translations-series-introduction?p=1
}
\end{displayquote}

\begin{displayquote}
While Joseph Smith and his contemporaries referred to his work on the Book of Abraham 
as a translation, Smith had no prior knowledge of the Egyptian language and relied 
instead on divine revelation to produce the text.\footnote{
https://www.josephsmithpapers.org/articles/book-of-abraham-and-related-manuscripts-volume-now-available
}
\end{displayquote}

If Joseph wasn't translating the Book of Abraham as we've been told, and it was
merely revelation, why would he need to translate an egyptian alphabet? Isn't that
wasting time?

\begin{displayquote}
July 1835

\textless Translating the Book of Abraham \&c.\textgreater The remainder of this month, I was 
continually engaged in translating an alphabet to the Book of Abraham, and 
arrangeing a grammar of the Egyptian language as practiced by the
ancients.\footnote{
https://www.josephsmithpapers.org/paper-summary/history-1838-1856-volume-b-1-1-september-1834-2-november-1838/51
}
\end{displayquote}