\section{Literal vs Symbolic}

The LDS Church is full of symbolism, the temple being one of the biggest symbols of
all. However the bible contains stories which can be considered literal, or symbolic
as well. A few to mention are Noah and the great flood, and Jonah swalled by the
whale (although I think the bible actually says it's a big fish).

So which is it? Are these stories factual? Are they symbolic in nature to be faith
promoting stories? What of the Book of Mormon, is it factual? Is it faith promoting?

There are creatures talked about in the bible which we consider mystical today. But
according to the Bible Dictionary, they are more of an ox. Yes, I'm talking about the
Unicorn.

Unicorns? Unicorns.

Here are the places in the bible where it mentions these creatures:

\begin{displayquote}
He maketh them also to skip like a calf; Lebanon and Sirion like a young 
unicorn.\footnote{Psalms 29:6}
\end{displayquote}

\begin{displayquote}
Will the unicorn be willing to serve thee, or abide by thy crib?\footnote{Job 39:9}
\end{displayquote}

\begin{displayquote}
Canst thou bind the unicorn with his band in the furrow? or will he harrow the 
valleys after thee?\footnote{Job 39:10} 
\end{displayquote}

\begin{displayquote}
God brought them out of Egypt; he hath as it were the strength of an 
unicorn.\footnote{Numbers 23:22}
\end{displayquote}

\begin{displayquote}
But my horn shalt thou exalt like the horn of an unicorn: I shall be 
anointed with fresh oil.\footnote{Psalms 92:10}
\end{displayquote}

\begin{displayquote}
God brought him forth out of Egypt; he hath as it were the strength of an 
unicorn: he shall eat up the nations his enemies, and shall break their 
bones, and pierce them through with his arrows.\footnote{Numbers 24:8}
\end{displayquote}

\begin{displayquote}
And the unicorns shall come down with them, and the bullocks with the bulls; and 
their land shall be soaked with blood, and their dust made fat with 
fatness.\footnote{Isaiah 34:7} 
\end{displayquote}

\begin{displayquote}
Save me from the lion’s mouth: for thou hast heard me from the 
horns of the unicorns.\footnote{Psalms 22:21}
\end{displayquote}

\begin{displayquote}
His glory is like the firstling of his bullock, and his horns are like the horns 
of unicorns: with them he shall push the people together to the ends of the 
earth: and they are the ten thousands of Ephraim, and they are the thousands 
of Manasseh.\footnote{Deuteronomy 33:17} 
\end{displayquote}

And then of course we have the Bible Dictionary description:

\begin{displayquote}
A wild ox, the \textit{Bos primigenius}, now extinct, but once common in Syria. 
The KJV rendering is unfortunate, as the animal intended is two-horned.\footnote{
LDS Bible Dictionary: Unicorn
}
\end{displayquote}

Scholars have indicated that it is not the actual mystical creature unicorn which we
have come to know and love. Which is kind of a disappointment. But if it can be in
there, and be misinterpreted in these days, what else might the bible be hiding?