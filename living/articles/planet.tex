\chapter{My Own Planet}

Again from the Mormon Newsroom article:

\begin{displayquote}
\textbf{Do Latter-day Saints believe that they will ``get their own planet"?}

No. This idea is not taught in Latter-day Saint scripture, nor is it a doctrine 
of the Church. This misunderstanding stems from speculative comments 
unreflective of scriptural doctrine. Mormons believe that we are all sons and 
daughters of God and that all of us have the potential to grow during and after 
this life to become like our Heavenly Father (see Romans 8:16-17). The Church 
does not and has never purported to fully understand the specifics of Christ’s 
statement that ``in my Father's house are many mansions" 
(John 14:2).\footnote{https://www.mormonnewsroom.org/article/mormonism-101}
\end{displayquote}

I remember being on my mission and people asked this question. We would say 
exactly what was stated above. Yet we knew, through the temple and other 
teachings, that it was possible to become a God and we would be creating 
spirit children and planets to put those spirit children on.

At least that's what we thought to be true. Yet here we are, another article
that states differently what was taught from before. So, again... I feel like
a broken record at this point, why the change?

At this rate, I feel like all I can ever become is a servent of God in the 
after life. That I'll never be able to enjoy the fullness of perfection and
explore everything that He has and is allowed to explore. To be taught 
these things from the beginning at a young age and then to find out they 
are changed? It's disconcerting to say the least. It almost feels like I've been
lied to. It almost feels like none of it matters anymore. Why bother with trying
to do anything in this life. Just keeping my nose clean seems to be the best
option at this point.

What exactly is there to strive for?

When I was younger, I recall thinking to myself: 

\begin{displayquote}
When I get to create a planet, I am going to populate it with 
penguins and palm trees. 
\end{displayquote}

Go ahead and laugh, that's what I thought. I thought it would be so cool to be 
able to create something like God had created. To be able to speak and have it 
organized just like in Genesis, Moses, and Abraham.

But I suppose that's no longer the case.

Now I can see some people saying, ``Oh, that's not what the church is saying
at all. They just don't want to give out meat before milk." Well, if that's
the case? Then the church is simply saying half truths which is in effect a
lie. God commanded ``Thou shalt not bear false witness against thy 
neighbour."\footnote{Exodus 20:16} Did he not? You know, the whole lying thing
is against God's will.

\begin{displayquote}
Wo unto the liar, for he shall be thrust down to hell.\footnote{2 Nephi 9:34}
\end{displayquote}

Naturally when questions about changes or other doctrine comes up that conflict
with what we've been taught in the past, or go against better judgmenet and
logic; we are told to have faith. Only believe. God will take care of everything
in the end and we don't need to worry about it right here and now. I suppose
that's fine for some, but to not have an idea of what's going to happen when
we get through with this life? That makes things difficult. If we're just 
going to be hanging out a celestial waiting room for eternity, yeah I'm not
sure how I would handle that.

There's an interesting thought, who's lying exactly? We are told that God can't
lie. It's impossible for Him to do so.\footnote{Hebrews 6:18}

Is changing what once was, lying? Not all changes can be chalked up to lying
right? But if it's not truth and it was taught as truth, what is it exactly?
Where does it fit in?

Being troubled by change is difficult. A consistant amount of belief is healthy
and reasonable for me. To have believed in one thing for so long, then to have
that narrative changed. It honestly feels like a rug has been ripped out from
under me.

We are told the wiseman built his house upon rocks, the foolishman built his
house upon sand.\footnote{Matthew 7:24-27}

The Gospel of Jesus Christ has been compared to a rock.\footnote{Figuratively, 
Jesus Christ and His gospel, which are a strong foundation and support 
(D\&C 11:24; 33:12–13). Rock can also refer to revelation, 
by which God makes His gospel known to man (Matt. 16:15–18).
[https://www.lds.org/scriptures/gs/rock?lang=eng]} If prophets and apostles
are changing the narrative of the Gospel of Jesus Christ, then where is the
rock upon which we can stand and be sure?