\documentclass{article}

\usepackage[backend=biber]{biblatex}
\usepackage{csquotes}
\bibliography{uni.bib}

\pagestyle{headings}

\title{Changing Times}
\author{Kyle Eggleston}

\begin{document}
\pagenumbering{gobble}
\maketitle
\newpage

\tableofcontents
\thispagestyle{empty}
\newpage

\pagenumbering{arabic}

\section{Things Change}

\begin{displayquote}
In the beginning was the word. The word was with God. The word was of 
God.\footnote{John 1:1} 
\end{displayquote}

Jesus Christ was the \textit{word} that is spoken of.

It could be said that Jesus was God. But only in the sense that he was the 
God of the Old Testament.\cite{otStudentManual} However, that's not how the
words were originally written:

\begin{displayquote}
In the beginning was the gospel preached through the Son. And the gospel was 
the word, and the word was with the Son, and the Son was with God, and the Son 
was of God.\footnote{JST John 1:1}
\end{displayquote}

So, why the change? We're told that many plain and precious truths of the bible 
had been removed and lost,\footnote{1 Nephi 13:26-27} from the \textit{``great 
and abominable church''}.

There was a council of Carthage (397) in which it was decided the books to
be considered canon for the Bible. The books were named and no books have
been added to the Bible since.

Who was this ``great and abominable church"?

That question is still up for debate. Obviously it's the church of the devil. 
But is there a church standing today that classifys as that? I'd rather not 
go into that. We know what Bruce R. McConkie thought about it in Mormon 
Doctrine, that theory had later changed, and was removed from the book 
completely. Mormon Doctrine is no longer in Deseret Bookstore shelves. I wonder 
why that is.\footnote{[Under the heading, ``Church of the Devil," Apostle Bruce 
R. McConkie lists:] "The Roman Catholic Church specifically—singled out, set 
apart, described, and designated as being ‘most abominable above all other 
churches’ (I Ne. 13:5)" (Mormon Doctrine, 1958, 129).}

With changing times come other changes that people make. I suppose this article 
is all about change isn't it.

People change as time changes. What was right back in the 1800s or the 1600s, 
isn't right now. Hanging a witch, for example, people got that from Exodus 
22:18.\footnote{Thou shalt not suffer a witch to live.} But, we learn from the 
Joseph Smith Translation of the Bible, that it's not the word 
witch.\footnote{Thou shalt not suffer a \textit{murderer} to live. 
[JST Exodus 22:18]} Quite different between a witch and a murderer right?

Changes happen because men polute the words of God.\footnote{Mormon 8:36,38}

So, why would God allow these changes? We are told He allows people to have
agency, yet one would think He wouldn't allow men to pollute His holy word?
I suppose it is neither here nor there. That is fine and well.

I should point out that not all changes are evil. Not all changes come from 
Satan, the Devil, the Father of all lies. There are some changes in life that 
are actually good. Some changes that come because change was needed. You can see 
it in history. If you don't know what kind of changes I'm talking about, 
seriously go crack open a history book and see all there is to see and learn 
about.

It should also be pointed out that I question at times if the fullness of the 
gospel was restored, why does there need to be change? Why wasn't it that way 
from the beginning? Herein, I shall go over some changes which have occurred
over the course of history.

If things need changing, I believe they shouldn't have been in their original
form to begin with. But again, that is my thought process on the matter.

\newpage

\section{Race and the Eternal Salvation Ban}

Long ago, the LDS Church stated that the negro race weren't allowed to hold the 
Holy Priesthood of God. They weren't able to attend the temple either. This 
lasted for several years. Then change came about, and in 1978 a revelation was 
passed down that lifted this ban.

Before the change, presidents and apostles of the church had no issue stating in 
no uncertain terms that the ban was of God. It was god's doing, the Lord put the 
ban in place and it was His purpose for doing so. That it was doctrine.

\begin{displayquote}
Ham, through Egyptus, continued the curse which was placed upon the seed of 
Cain. Because of that curse this dark race was separated and isolated from all 
the rest of Adam's posterity before the flood, and since that time the same 
condition has continued, and they have been `despised among all people.' 
This \textbf{doctrine} did not originate with President Brigham Young but was 
taught by the Prophet Joseph Smith ... we all know it is due to his teachings 
that the negro today is barred from the 
Priesthood.\footnote{The Way to Perfection, pages 110-111}
\end{displayquote}

However, according to the Gospel Topic Essays, we learn:

\begin{displayquote}
Over time, Church leaders and members advanced many theories to explain the 
priesthood and temple restrictions. None of these explanations is accepted 
today as the official doctrine of the 
Church.\footnote{Race and the Priesthood, LDS.org}
\end{displayquote}

Others, like John Taylor, taught more ... unsettling things:

\begin{displayquote}
And after the flood we are told that the curse that had been pronounced upon 
Cain was continued through Ham's wife, as he had married a wife of that seed. 
And why did it pass through the flood? because it was necessary that the devil 
should have a representation upon the earth as well as 
God ...\footnote{Journal of Discourses, 22:304}
\end{displayquote}

This was focused on more than once:

\begin{displayquote}
Why is it, in fact, that we should have a devil? Why did the Lord not kill 
him long ago? Because he could not do without him. He needed the devil and a 
great many of those who do his bidding to keep men straight, that we may learn 
to place our dependence on God, and trust in Him, and to observe his laws and 
keep his commandments. When he destroyed the inhabitants of the antediluvian 
world, he suffered a descendant of Cain to come through the flood in order 
that he might be properly represented upon the 
earth.\footnote{Journal of Discourses, 23:336}
\end{displayquote}

Then there were these quotes:

\begin{displayquote}
Shall I tell you the law of God in regard to the African race? If the white 
man who belongs to the chosen seed mixes his blood with the seed of Cain, 
the penalty, under the law of God, is death on the spot. This will 
always be so.\footnote{Brigham Young, Journal of Discourses, Vol 10, page 110}
\end{displayquote}

Some taught that it was of God and was the Lord's doing:

\begin{displayquote}
Negroes in this life are denied the Priesthood; under no circumstances can 
they hold this delegation of authority from the Almighty. (Abra. 1:20-27.) 
The gospel message of salvation is not carried affirmatively to them... 
negroes are not equal with other races where the receipt of certain 
spiritual blessings are concerned, particularly the priesthood and 
the temple blessings that flow there from, but this inequality is 
not of man's origin. It is the Lord's doing, is based on his eternal 
laws of justice, and grows out of the lack of Spiritual valiance of those 
concerned in their first estate.\footnote{Mormon Doctrine, 1966, pp. 527-528}
\end{displayquote}

There are several more I could include in this paper, but I believe these
are sufficient for now.

Personally, reading such quotes turns my stomach. I do not understand how a 
prophet of God could speak like that. If we are truely to love our brothers
and sisters as Christ taught, one would think that these teachings wouldn't
have occurred.

After the ban, they said it was folklore. The reasons for doing so was because 
of man.

How is it one generation of prophets and apostles can call an older generation 
false on their teachings? Teachings people followed because they were following 
the prophet?

Brigham Young stated that he was afraid people would have too much faith in the 
presidency of the church and him as a prophet that they wouldn't ask God if 
something was right.\footnote{I am more afraid that this people have so much 
confidence in their leaders that they will not inquire for themselves of God 
whether they are lead by him. [Brigham Young, (12 January 1862) Journal of 
Discourses 9:150]}

Well, if you were all for a black person getting the priesthood, or questioned 
your leaders about it because you felt that was the correct course of 
action...you were facing excommunication.

So change can come, but it can also come at quite a price.

I think it is okay to ask this. If the ban wasn't of God as church leaders are 
now saying, then why did God allow it? Why would God allow such a thing to take 
place? If it was indeed ``folklore", one would think God wouldn't allow prophets 
and apostles of the church to allow people to think the negro would never 
receive the priesthood and temple ordinances.

There are so many quotes on the matter it sickens me to think about it.

But it is all cleared up by one remark by an apostle. Bruce R. McConkie:

\begin{displayquote}
Forget everything that I have said, or what President Brigham Young or 
President George Q. Cannon or whomsoever has said in days past that is contrary 
to the present revelation. We spoke with a limited understanding and without the 
light and knowledge that now has come into the world.\footnote{All Are Alike 
Unto God, Bruce R. McConkie, Aug 18, 1978}
\end{displayquote}

In The Deseret News, we find a quote from Jeffry R. Holland:

\begin{displayquote}
Likewise, the current leadership of the church has spoken on the need to 
abandon the racist teachings that long circulated within Mormonism 
regarding the ban. Elder Jeffery R. Holland, a current member of the 
Council of the Twelve, recently said in a public interview 
``One clear-cut position is that the folklore must never be perpetuated...
I think almost all of (these teachings) were inadequate and/or 
wrong."\footnote{Deseret News, Race, folklore and Mormon doctrine, 
Nathan B. Oman, February 29, 2012}
\end{displayquote}

If change can be simply accepted based on a revelation from God then that is 
good right? Why did it take a revelation to change policy? The church claims it 
was a policy not doctrine. Even though it was taught as doctrine throughout the 
course of history.

Do the lines between doctrine and policy blur at times? Perhaps more change?

There are scriptures that reference to the people's skin being turned dark due 
to sin or not following God's will while on the Earth. Cain was the first man to 
go dark because of murder. A mark of darkness was placed upon man in the event 
that anyone would come across him.\footnote{And the Lord said unto him, 
Therefore whosoever slayeth Cain, vengeance shall be taken on him sevenfold. 
And the Lord set a mark upon Cain, lest any finding him should kill 
him.[Genesis 4:15]}

In the Book of Mormon we learn about the Lamenites and the Nephites. The 
Lamenites had the dark skin:

\begin{displayquote}
And the skins of the Lamanites were dark, according to the mark which was set 
upon their fathers, which was a curse upon them because of their transgression 
and their rebellion against their brethren, who consisted of Nephi, Jacob, and 
Joseph, and Sam, who were just and holy men.\footnote{Alma 3:6}
\end{displayquote}

God says that the cursing is so the wicked people wouldn't be enticing to those
who followed the commandments of God.

\begin{displayquote}
And he had caused the cursing to come upon them, yea, even a sore cursing, 
because of their iniquity. For behold, they had hardened their hearts against 
him, that they had become like unto a flint; wherefore, as they were white, 
and exceedingly fair and delightsome, that they might not be enticing unto my 
people the Lord God did cause a skin of blackness to come upon 
them.\footnote{2 Nephi 5:21}
\end{displayquote}

Before 1978, there was certain things taught. One of those teachings was that 
the dark skinned people were less valiant in the pre-existence. This has been
shot down. There were no fence sitters in the pre-existence in the war in 
heaven. Either you chose Jesus or you chose Lucifer.\footnote{Apostle Joseph 
Fielding Smith, for example, wrote in 1907 that the belief was ``quite general" 
among Mormons that ``the Negro race has been cursed for taking a neutral 
position in that great contest." Yet this belief, he admitted, ``is not the 
official position of the Church, [and is] merely the opinion of men." 
Joseph Fielding Smith to Alfred M. Nelson, Jan. 31, 1907, 
Church History Library, Salt Lake City.}

Now you'll notice I called this an ``Eternal Salvation Ban" not simplay a 
``Priesthood Ban" as the church tends to simplify it. No, it's more than that.
It was a temple ban. People of color weren't able to be sealed to their loved
ones, which is one of the main points of LDS Doctrine. The idea of eternal
families.

Feels like a slap to the face of those wanting to be sealed to their spouses,
children, parents, loved ones etc. If you claim to have revelation from God and
part of that is that the whole human race has the ability to be together 
forever, why would God allow for man to withold that from his children?

Now, the church considers it a revelation. But in an interview with the apostle
LeGrand Richards, it sounds quite different.

\begin{displayquote}
WALTERS: Now when President Kimball read this little announcement or paper, 
was that the same thing that was released to the press?

RICHARDS: Yes.

WALTERS: There wasn't a special document as a ``revelation", that he had and 
wrote down?

RICHARDS: We discussed it in our meeting. What else should we say besides 
that announcement? And we decided that was sufficient; that no more 
needed to be said.\footnote{Interview with Apostle LeGrand Richards,
By Wesley P. Walters and Chris Vlachos, 16th August 1978, Church Office Building
(Recorded on Cassette)}
\end{displayquote}

There was no ``Thus saith the Lord" in the Official Declaration 2. So I question
you, dear reader, was it a revelation? I dare say it wasn't. I dare say it was
a policy change. I dare say what was once taught as doctrine and taught as it
was from God was changed by the pressures and will of man.

Speaking of Official Declaration 2, here is the text in its entirety.

\begin{displayquote}
To Whom It May Concern:

On September 30, 1978, at the 148th Semiannual General Conference of The 
Church of Jesus Christ of Latter-day Saints, the following was presented by 
President N. Eldon Tanner, First Counselor in the First Presidency of the 
Church:

In early June of this year, the First Presidency announced that a revelation 
had been received by President Spencer W. Kimball extending priesthood and 
temple blessings to all worthy male members of the Church. President Kimball 
has asked that I advise the conference that after he had received this 
revelation, which came to him after extended meditation and prayer in the 
sacred rooms of the holy temple, he presented it to his counselors, who 
accepted it and approved it. It was then presented to the Quorum of the 
Twelve Apostles, who unanimously approved it, and was subsequently presented 
to all other General Authorities, who likewise approved it unanimously.

President Kimball has asked that I now read this letter:

June 8, 1978

To all general and local priesthood officers of The Church of Jesus Christ 
of Latter-day Saints throughout the world:

Dear Brethren:

As we have witnessed the expansion of the work of the Lord over the earth, we 
have been grateful that people of many nations have responded to the message 
of the restored gospel, and have joined the Church in ever-increasing numbers. 
This, in turn, has inspired us with a desire to extend to every worthy member 
of the Church all of the privileges and blessings which the gospel affords.

Aware of the promises made by the prophets and presidents of the Church who have 
preceded us that at some time, in God’s eternal plan, all of our brethren who 
are worthy may receive the priesthood, and witnessing the faithfulness of those 
from whom the priesthood has been withheld, we have pleaded long and earnestly 
in behalf of these, our faithful brethren, spending many hours in the Upper Room 
of the Temple supplicating the Lord for divine guidance.

He has heard our prayers, and by revelation has confirmed that the long-promised 
day has come when every faithful, worthy man in the Church may receive the holy 
priesthood, with power to exercise its divine authority, and enjoy with his 
loved ones every blessing that flows there from, including the blessings of the 
temple. Accordingly, all worthy male members of the Church may be ordained to 
the priesthood without regard for race or color. Priesthood leaders are 
instructed to follow the policy of carefully interviewing all candidates for 
ordination to either the Aaronic or the Melchizedek Priesthood to insure that 
they meet the established standards for worthiness.

We declare with soberness that the Lord has now made known his will for the 
blessing of all his children throughout the earth who will hearken to the 
voice of his authorized servants, and prepare themselves to receive every 
blessing of the gospel.

Sincerely yours,

SPENCER W. KIMBALL

N. ELDON TANNER

MARION G. ROMNEY

The First Presidency

Recognizing Spencer W. Kimball as the prophet, seer, and revelator, and 
president of The Church of Jesus Christ of Latter-day Saints, it is proposed 
that we as a constituent assembly accept this revelation as the word and 
will of the Lord. All in favor please signify by raising your right hand. 
Any opposed by the same sign.

The vote to sustain the foregoing motion was unanimous in the affirmative.

Salt Lake City, Utah, September 30, 1978.\footnote{Official Declaration 2, 
Doctrine and Covenants}
\end{displayquote}

It is a wonderful thing that this ban was lifted. It is a shame it ever was in
place to begin with. Imagine all of those years of racism and hatred that could
have been done without. People believed God spoke and they followed Him. The
prophet led them and he couldn't be wrong...even when he was saying that those
under the ban would never receive the priesthood in this life.

If they were speaking as men, which I truely hope they were, why would God allow
such a thing? Why would He allow such teachings to go on for so many years? I 
ask it all again. Why?

There are many things in this life that don't add up or make sense. I suppose
this is one of them. To understand it in another life, to have to wait to be 
able to understand it in another life? Why would that be? It would seem with
the changing narrative, dismissing those who have spoken ``as prophets of God",
seems to downplay it all. The church doesn't want to come off as racist. That
is understandable. But instead of brushing it under a rug, why not apologize?

Was there ever a full formal apology regarding it? Or was this new ``revelation"
simply all there was to make things better? It feels like they put a band-aid
over a wound simply to let it heal and go away eventually.

The interesting thing about history, it doesn't just go away. Those teachings
of former prophets are still around. With the internet and this day in age,
those teachings will never be lost. No matter how much people wish it would go
away, it will never be lost. People will always be able to find it, research it,
and learn what happened and form an opinion on it; after they have read all of
the facts.

\newpage

\section{As God now is, man may be}

As was taught from teachings of a certain president of the Church of Jesus
Christ of Latter-day Saints, Lorenzo Snow taught:

\begin{displayquote}
As man now is, God once was:

As God now is, man may be.\footnote{In Eliza R. Snow Smith, Biography and 
Family Record of Lorenzo Snow (1884), 46; see also ``The Grand Destiny of Man," 
Deseret Evening News, July 20, 1901, 22.}
\end{displayquote}

This appears to be another thing has has gone under some change? For according
to the Church of Jesus Christ of Latter-day Saint's official news 
page,\footnote{http://mormonnewsroom.com} we don't follow that teaching anymore.

Let's pull a quote directly from a FAQ on that site:

\begin{displayquote}
\textbf{Do Latter-day Saints believe they can become ``gods"?}

Latter-day Saints believe that God wants us to become like Him. But this 
teaching is often misrepresented by those who caricature the faith. 
The Latter-day Saint belief is no different than the biblical teaching, 
which states, ``The Spirit itself beareth witness with our spirit, 
that we are the children of God: and if children, then heirs; heirs of God, 
and joint-heirs with Christ; if so be that we suffer with him, that we may be 
also glorified together" (Romans 8:16-17). Through following Christ's 
teachings, Latter-day Saints believe all people can become ``partakers of the 
divine nature"
(2 Peter 1:4).\footnote{https://www.mormonnewsroom.org/article/mormonism-101}
\end{displayquote}

If church members are not taught that we can become Gods, what was the 
revelation in Doctrine and Covenants 76 for? It teaches of the three kingdoms
of God, specifically the Celestial, Terrestrial, and Telestial kingdoms.

There's a scripture in that, verse 58 that states:

\begin{displayquote}
Wherefore, as it is written, they are gods, even the sons of 
God—\footnote{D\&C 76:58}
\end{displayquote}

Then there's the scripture in section 132:

\begin{displayquote}
And again, verily I say unto you, if a man marry a wife by my 
word, which is my law, and by the new and everlasting covenant, 
and it is sealed unto them by the Holy Spirit of promise, by him 
who is anointed, unto whom I have appointed this power and the keys
of this priesthood; and it shall be said unto them—Ye shall come 
forth in the first resurrection; and if it be after the first 
resurrection, in the next resurrection; and shall inherit thrones, 
kingdoms, principalities, and powers, dominions, all heights and 
depths—then shall it be written in the Lamb’s Book of Life, that 
he shall commit no murder whereby to shed innocent blood, and if 
ye abide in my covenant, and commit no murder whereby to shed innocent 
blood, it shall be done unto them in all things whatsoever my servant 
hath put upon them, in time, and through all eternity; and shall be of 
full force when they are out of the world; and they shall pass by the 
angels, and the gods, which are set there, to their exaltation and 
glory in all things, as hath been sealed upon their heads, which 
glory shall be a fulness and a continuation of the seeds 
forever and ever.

Then shall they be gods, because they have no end; therefore shall 
they be from everlasting to everlasting, because they continue; then 
shall they be above all, because all things are subject unto them. 
Then shall they be gods, because they have all power, and the 
angels are subject unto them.\footnote{D\&C 132:19-20}
\end{displayquote}

This is describing those who belong to the Celestial Kingdom. If we are not to
become Gods, as is stated in the Mormon Newsroom article, then what is it? Which
source does one believe pertaining to their eternal salvation, given that they
``come forth in the resurrection of the just."\footnote{D\&C 76:50} 

Past prophets speaking vs current policy teaching. Which is true and which is
false? Again, why a change? Why can't the church stand boldly in what they have
taught to be the truth and continue with it? Why must changes need to be made?

If God is the same yesterday, today, and forever why does He change? Is it
simply because times change? It is taught that God must follow the laws of
science and the other material laws when it comes to creation etc., yet if He
changes things now, or allows men to change things, how are we supposed to know
He won't change things after we have died?

\begin{displayquote}
For do we not read that God is the same yesterday, today, and forever, and 
in him there is no variableness neither shadow of 
changing?

And now, if ye have imagined up unto yourselves a god who doth vary, and in 
whom there is shadow of changing, then have ye imagined up unto yourselves a 
god who is not a God of miracles.\footnote{Book of Mormon 9:9}
\end{displayquote}

So, which is it? Is God a God of mircales? Or is He changing as the times here
on earth see fit?

In the book Gospel Principles, in a chapter on Exaltation, it once said:

\begin{displayquote}
\textbf{WHAT IS EXALTATION?}

Exaltation is eternal life, the kind of life that God lives. He lives in great
glory. He is perfect. He possesses all knowledge and all wisdom. He is the
father of spirit children. He is a creator. We can become Gods like our Heavnly
Father. This is exaltation.

If we prove faithful and obedient to all the commandments of the Lord, we will
live in the highest degree of the celestial kingdom of heaven. We will become
exalted, just like our Heavenly Father. Exaltation is the highest reward that
our Heavenly Father can give his children. The Lord has said that exaltation
is the greatest gift of all the gifts of 
God (see D\&C 14:7).\cite[pp. 289-290]{gp}
\end{displayquote}

That text was from a 1979 revised edition of the book, originally recommended
to missionaries as part of the Missionary Reference Library. I carred it on 
my mission and have access to the book. When compared to a later version, 
the narriative has changed. I will put an elipses in to show where the 
change is:

\begin{displayquote}
\textbf{What is exaltation?}

Exaltation is eternal life, the kind of life God lives. He lives in great glory. 
He is perfect. He possesses all knowledge and all wisdom. He is the 
Father of spirit children. He is a creator. We can become [...] like our 
Heavenly Father. This is exaltation.

If we prove faithful to the Lord, we will live in the highest degree of the 
celestial kingdom of heaven. We will become exalted, to live with our 
Heavenly Father in eternal families. Exaltation is the greatest gift that 
Heavenly Father can give His children (see D\&C 14:7).\cite[275-280]{gp2}
\end{displayquote}

You'll notice they took out the word Gods in that first paragraph. It has 
changed from telling us that we can become Gods to just that we can become
like our Heavenly Father. No promise of Godhood there.

The second paragraph, well you can see the change for yourself. I believe it 
speaks for itself quite well.

So, what brings about such changes? They were fine for earlier members of the
church. Why would they be changed now? It should be considered a doctrinal 
change. The emphasis has been changed over the years to show living with God 
in the post-mortal life, instead of becoming Gods ourselves.

I find a lot of the older doctrine as it were isn't taught much in these much 
later days. I wonder why that is. Are they too being tossed aside as people 
speaking as a man? I would doubt so. It is interesting that no one has spoken 
much in General Conference of the King Follett sermon lately.

\begin{displayquote}
God himself was once as we are now, and is an exalted man, and sits enthroned 
in yonder heavens! That is the great secret. If the veil were rent today, 
and the great God who holds this world in its orbit, and who upholds all 
worlds and all things by His power, was to make himself visible—I say, if 
you were to see him today, you would see him like a man in form—like yourselves 
in all the person, image, and very form as a man; for Adam was created in the 
very fashion, image and likeness of God, and received instruction from, and 
walked, talked and conversed with Him, as one man talks and communes with 
another.\footnote{King Follett Sermon, Joseph Smith Jr.}
\end{displayquote}

In it we are taught that God was once a man, which is the first part of Snow's 
couplet. Yet this is not openly widely taught these days. So yet another change 
has easily taken place. This is not to say it is not known, for the text is out 
there to be found. But it is not actively taught.

\newpage

\section{My Own Planet}

Again from the Mormon Newsroom article:

\begin{displayquote}
\textbf{Do Latter-day Saints believe that they will ``get their own planet"?}

No. This idea is not taught in Latter-day Saint scripture, nor is it a doctrine 
of the Church. This misunderstanding stems from speculative comments 
unreflective of scriptural doctrine. Mormons believe that we are all sons and 
daughters of God and that all of us have the potential to grow during and after 
this life to become like our Heavenly Father (see Romans 8:16-17). The Church 
does not and has never purported to fully understand the specifics of Christ’s 
statement that ``in my Father's house are many mansions" 
(John 14:2).\footnote{https://www.mormonnewsroom.org/article/mormonism-101}
\end{displayquote}

I remember being on my mission and people asked this question. We would say 
exactly what was stated above. Yet we knew, through the temple and other 
teachings, that it was possible to become a God and we would be creating 
spirit children and planets to put those spirit children on.

At least that's what we thought to be true. Yet here we are, another article
that states differently what was taught from before. So, again... I feel like
a broken record at this point, why the change?

At this rate, I feel like all I can ever become is a servent of God in the 
after life. That I'll never be able to enjoy the fullness of perfection and
explore everything that He has and is allowed to explore. To be taught 
these things from the beginning at a young age and then to find out they 
are changed? It's disconcerting to say the least. It almost feels like I've been
lied to. It almost feels like none of it matters anymore. Why bother with trying
to do anything in this life. Just keeping my nose clean seems to be the best
option at this point.

What exactly is there to strive for?

When I was younger, I recall thinking to myself: 

\begin{displayquote}
When I get to create a planet, I am going to populate it with 
penguins and palm trees. 
\end{displayquote}

Go ahead and laugh, that's what I thought. I thought it would be so cool to be 
able to create something like God had created. To be able to speak and have it 
organized just like in Genesis, Moses, and Abraham.

But I suppose that's no longer the case.

Now I can see some people saying, ``Oh, that's not what the church is saying
at all. They just don't want to give out meat before milk." Well, if that's
the case? Then the church is simply saying half truths which is in effect a
lie. God commanded ``Thou shalt not bear false witness against thy 
neighbour."\footnote{Exodus 20:16} Did he not? You know, the whole lying thing
is against God's will.

\begin{displayquote}
Wo unto the liar, for he shall be thrust down to hell.\footnote{2 Nephi 9:34}
\end{displayquote}

Naturally when questions about changes or other doctrine comes up that conflict
with what we've been taught in the past, or go against better judgmenet and
logic; we are told to have faith. Only believe. God will take care of everything
in the end and we don't need to worry about it right here and now. I suppose
that's fine for some, but to not have an idea of what's going to happen when
we get through with this life? That makes things difficult. If we're just 
going to be hanging out a celestial waiting room for eternity, yeah I'm not
sure how I would handle that.

There's an interesting thought, who's lying exactly? We are told that God can't
lie. It's impossible for Him to do so.\footnote{Hebrews 6:18}

Is changing what once was, lying? Not all changes can be chalked up to lying
right? But if it's not truth and it was taught as truth, what is it exactly?
Where does it fit in?

Being troubled by change is difficult. A consistant amount of belief is healthy
and reasonable for me. To have believed in one thing for so long, then to have
that narrative changed. It honestly feels like a rug has been ripped out from
under me.

We are told the wiseman built his house upon rocks, the foolishman built his
house upon sand.\footnote{Matthew 7:24-27}

The Gospel of Jesus Christ has been compared to a rock.\footnote{Figuratively, 
Jesus Christ and His gospel, which are a strong foundation and support 
(D\&C 11:24; 33:12–13). Rock can also refer to revelation, 
by which God makes His gospel known to man (Matt. 16:15–18).
[https://www.lds.org/scriptures/gs/rock?lang=eng]} If prophets and apostles
are changing the narrative of the Gospel of Jesus Christ, then where is the
rock upon which we can stand and be sure?

\newpage

\printbibliography
\thispagestyle{empty}

\end{document}
