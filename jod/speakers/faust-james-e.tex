\section{James E. Faust}

\subsection{The Prophetic Voice, April 1996}

The dispensation of divine truth in which we now live, in distinction from previous dispensations, will not be destroyed by apostasy. This is in fulfillment of Daniel’s prophecy that ``the God of heaven would set up a kingdom, which shall never be destroyed" nor ``left to other people." (Dan. 2:44; see also D\&C 138:44) President John Taylor affirmed this also when he said: ``There is one thing very certain, ... and that is, whatever men may think, and however they may plot and contrive, that this Kingdom will never be given into the hands of another people. It will grow and spread and increase, and no man living can stop its progress." (In Journal of Discourses, 25:348\footnote{There is one thing very certain, very certain indeed, and that is, whatever men may think, and however they may plot and contrive, that this Kingdom will never be given into the hands of another people. It will grow and spread and increase, and no man living can stop its progress. Hence I feel quite easy, as I said before, for the Lord reigns, and let the people rejoice.}; see also 14:367\footnote{I will tell you the only thing I am afraid of about the Saints is that they will forget their God and that they will not live their religion; then again I have not that fear, because I know the generality of them will. I know this kingdom will not be given into the hands of another people. I know that it will continue to progress and continue to increase in spite of all the powers of the adversary, in spite of every influence that exists now, or that ever will exist on the face of this wide earth. God is our God, and he will bring off Israel triumphant.})

In an early revelation the Lord warned Oliver: ``Behold, thou art blessed, and art under no condemnation. But beware of pride, lest thou shouldst enter into temptation." (D\&C 23:1) Oliver had great intellect and enjoyed marvelous spiritual blessings. However, over time he forgot the Lord’s warning, and pride entered into his heart. Brigham Young later said of this pride: ``I have seen men who belonged to this kingdom, and who really thought that if they were not associated with it, it could not progress. One man especially, whom I now think of, ...\footnote{who} was peculiarly gifted in self-reliance and general ability. He said as much to the Prophet Joseph a number of times as to say that if he left this kingdom, it could not progress any further. I speak of Oliver Cowdery. He forsook it, and it still rolled on, and still triumphed over every opposing foe, and bore off safely all those who clung to it." (In Journal of Discourses, 11:252)

After being rebaptized, Thomas came to Salt Lake City, where he asked Brigham Young, the President of the Church, for forgiveness. He was invited by President Young to speak at a Sunday service where Thomas offered this advice to his listeners: ``If there are any among this people who should ever apostatize and do as I have done, prepare your backs for a good whipping, if you are such as the Lord loves. But if you will take my advice, you will stand by the authorities." (In Journal of Discourses, 5:206)

\subsection{Continuous Revelation, October 1989}

I wish to speak today of a special dimension of the gospel: the necessity for constant communication with God through the process known as divine revelation. This principle is basic to our belief. President Wilford Woodruff declared: ``Whenever the Lord had a people on the earth that He acknowledged as such, that people were led by revelation" (In Journal of Discourses, 24:240.) I affirm at the beginning that the inspiration of God is available to all who worthily seek the guidance of the Holy Spirit. This is particularly true of those who have received the gift of the Holy Ghost.

This process of continuous revelation comes to the Church very frequently. President Wilford Woodruff stated, ``This power is in the bosom of Almighty God, and he imparts it to his servants the prophets as they stand in need of it day by day to build up Zion." (in Journal of Discourses, 14:33.) This is necessary for the Church to fulfill its mission. Without it, we would fail.

This is in harmony with the counsel of Brigham Young: ``I am more afraid that this people have so much confidence in their leaders that they will not inquire for themselves of God whether they are led by Him. I am fearful they settle down in a state of blind self-security, trusting their eternal destiny in the hands of their leaders with a reckless confidence that in itself would thwart the purposes of God in their salvation, and weaken that influence they could give to their leaders, did they know for themselves, by the revelations of Jesus, that they are led in the right way. Let every man and woman know, by the whispering of the Spirit of God to themselves, whether their leaders are walking in the path the Lord dictates, or not." (In Journal of Discourses, 9:150.)

\subsection{The Magnificent Vision Near Palmyra, April 1984}

In the accounts of the Prophet and his mother, Lucy Mack Smith, there is also considerable historical background which has been confirmed by secondary sources as being accurate. As an example, the Prophet refers in the published account of the First Vision to the religious fervor in the area where the Smith family was living at the time. Among others, Brigham Young later affirmed: ``I very well recollect the reformation which took place in the country among the various denominations of Christians—the Baptists, Methodists, Presbyterians, and others—when Joseph was a boy." (Journal of Discourses, 12:67.)

\subsection{Integrity, the Mother of Many Virtues, April 1982}

Brigham Young said, ``If the Lord ever revealed anything to me, he has shown me that the Elders of Israel must let speculation alone and attend to the duties of their calling." (Journal of Discourses, 8:179.)

\subsection{Will I Be Happy?, April 1987}

Troubled as many homes may be in our society, we cannot abandon the home as the primary teacher of moral values. Nowhere else will moral values be taught so effectively. As Brigham Young counseled, we must teach children ``by faith rather than by the rod, leading them kindly by good example into all truth and holiness" (Journal of Discourses, 12:174).

\subsection{The Greatest Challenge in the World—Good Parenting, October 1990}

One of the most difficult parental challenges is to appropriately discipline children. Child rearing is so individualistic. Every child is different and unique. What works with one may not work with another. I do not know who is wise enough to say what discipline is too harsh or what is too lenient except the parents of the children themselves, who love them most. It is a matter of prayerful discernment for the parents. Certainly the overarching and undergirding principle is that the discipline of children must be motivated more by love than by punishment. Brigham Young counseled, ``If you are ever called upon to chasten a person, never chasten beyond the balm you have within you to bind up." (In Journal of Discourses, 9:124–25.) Direction and discipline are, however, certainly an indispensable part of child rearing. If parents do not discipline their children, then the public will discipline them in a way the parents do not like. Without discipline, children will not respect either the rules of the home or of society.

\subsection{Unwanted Messages, October 1986}

Do some of us seek to justify our taking of shortcuts and advantage of others by indulging in the twin sophistries, ``There isn't any justice" and ``Everybody does it"? There are many others who seemingly prosper by violating the rules of God and the standards of decency and fair play. They appear to escape the imminent law of the harvest, which states, ``Whatsoever a man soweth, that shall he also reap" (Gal. 6:7). Worrying about the punishment we think ought to come to others is self-defeating to us. Brigham Young counseled that unless we ourselves are prepared for the day of the Lord's vengeance when the wicked will be consumed, we should not be too anxious for the Lord to hasten his work. Said he rather, ``Let our anxiety be centered upon this one thing, the sanctification of our own hearts, the purifying of our own affections" (in Journal of Discourses, 9:3).