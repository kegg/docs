\section{Neal A. Maxwell}

\subsection{My Servant Joseph, April 1992}

Consecrated Joseph gave so much, yet often so little was returned. President Brigham Young lamented, ``There was confidence due from his brethren to Joseph which he did not receive. In his death they learned a profitable lesson, and afterwards felt that if he could only be restored to them how obedient they would be to his counsels." (In Journal of Discourses, 10:222.)

Joseph, the revelator. He also became an articulator. President Young said the Prophet Joseph had the ``happy faculty" of communicating things ``often in a single sentence throwing ...\footnote{a flood of} light into the gloom of [the] ages ...\footnote{He had power to draw the spirits of the people who listened to him to his standard, where they communed with heavenly objects and heavenly principles, connecting the heavenly and the earthly together—} in one blending flood of heavenly intelligence." (Brigham Young, in Journal of Discourses, 9:310.)

Joseph Smith lit up life's landscape, brethren, so that we can see ``things as they really are, and ... really will be." (Jacob 4:13.) The revelations about the dispensations in salvational history tell us that Adam had the fulness of Christ's gospel and all its ordinances. (See Moses 5:58–59.) Hence, Christianity did not begin with Jesus' mortal Messiahship in the meridian of time in Jerusalem! The diffusion which followed Adam naturally resulted in some similarities in various religions. Therefore, as President Joseph F. Smith declared, we find ``relics of Christianity" which ``date back ...\footnote{beyond the days of Moses, and even} beyond the flood, independent of ...\footnote{and apart from the records of} the Bible." (Journal of Discourses, 15:325.) Latter-day Saints are therefore unsurprised but instead are enriched whenever discoveries are made which show how the Lord grants ``unto all nations" to teach a portion of ``his word." (Alma 29:8.)

Brigham Young was not easily impressed by anybody, yet he said he felt like shouting ``Hallelujah!" all the time that he ever knew Joseph Smith! (See Journal of Discourses, 3:51.) And dying Brigham's last words were, ``Joseph! Joseph! Joseph!" He was about to be with his beloved Joseph once again! (Leonard J. Arrington, Brigham Young: American Moses, New York: Alfred A. Knopf, 1985, p. 399.)

Joseph could not have accomplished what he did if he had not become consecrated and spiritually submissive. Elder Erastus Snow warned the rest of us that when we are ``inclined to be stiff and refractory, ...\footnote{and we desire continually the gratification of our own will to the extent that this feeling prevails in us,} the Spirit of the Lord is held at a distance from us" because we are too busy gratifying our own wills, and thus we ``interpose a barrier" between us and God. (In Journal of Discourses, 7:352.)

Near the end, in multiple meetings, Joseph transferred the keys, authority, and ordinances to the Twelve. On one such occasion, President Wilford Woodruff said the revelator's ``face was as clear as amber and he was covered with a power [I have] never seen in [an instant] in the flesh before." (Wilford Woodruff, ``Journal History," 12 Mar. 1897.) President Young said that those who knew Joseph could tell when ``the Spirit of revelation was upon him, ...\footnote{for his countenance wore an expression peculiar to himself while under that influence. He preached by the Spirit of revelation, and taught in his council by it, and those who were acquainted with him could discover it at once,} for at such times there was a peculiar clearness and transparency in his face." (In Journal of Discourses, 9:89.)

Even with all he revealed, however, the Prophet Joseph knew much more than he could tell. President John Taylor observed that Joseph ``felt fettered and bound." (Journal of Discourses, 10:147–48.) Heber C. Kimball confirmed that Joseph sometimes felt ``as though he were enclosed ...,\footnote{in an iron case, his mind was closed by the influences that were thrown around him; he was curtailed in his wishes and desires to do good;} there was no room for him to expand, ...\footnote{hence he could not make use of the revelations of God as he would have done; there was} no room in the hearts of the people to receive.\footnote{the glorious truths
of the Gospel that God revealed to him.}" (In Journal of Discourses, 10:233.)

\subsection{According to the Desire of [Our] Hearts, October 1996}

Righteous desires need to be relentless, therefore, because, said President Brigham Young, ``the men and women, who desire to obtain seats in the celestial kingdom, will find that they must battle every day" (in Journal of Discourses, 11:14). Therefore, true Christian soldiers are more than weekend warriors.

Thus educating and training our desires clearly requires understanding the truths of the gospel, yet even more is involved. President Brigham Young confirmed, saying, ``It is evident that many who understand the truth do not govern themselves by it; consequently, no matter how true and beautiful truth is, you have to take the passions of the people and mould them to the law of God" (in Journal of Discourses, 7:55).

``Do you," President Young asked, ``think that people will obey the truth because it is true, unless they love it? No, they will not" (in Journal of Discourses, 7:55). Thus knowing gospel truths and doctrines is profoundly important, but we must also come to love them. When we love them, they will move us and help our desires and outward works to become more holy.

Fortunately for us, our loving Lord will work with us, ``even if [we] can [do] no more than desire to believe," providing we will ``let this desire work in [us]" (Alma 32:27). Therefore, declared President Joseph F. Smith, ``the education then of our desires is one of far-reaching importance to our happiness in life" (Gospel Doctrine, 5th ed. [1939], 297). Such education can lead to sanctification until, said President Brigham Young, ``holy desires produce corresponding outward works" (in Journal of Discourses, 6:170). Only by educating and training our desires can they become our allies instead of our enemies!

\subsection{Swallowed Up in the Will of the Father, October 1995}

As one's will is increasingly submissive to the will of God, he can receive inspiration and revelation so much needed to help meet the trials of life. In the trying and very defining Isaac episode, faithful Abraham ``staggered not ... through unbelief" (Rom. 4:20). Of that episode John Taylor observed that ``nothing but the spirit of revelation could have given him this confidence, and ...\footnote{and it was that which} sustained him under these peculiar circumstances" (in Journal of Discourses, 14:361). Will we too trust the Lord amid a perplexing trial for which we have no easy explanation? Do we understand—really comprehend—that Jesus knows and understands when we are stressed and perplexed? The complete consecration which effected the Atonement ensured Jesus' perfect empathy; He felt our very pains and afflictions before we did and knows how to succor us (see Alma 7:11–12; 2 Ne. 9:21). Since the Most Innocent suffered the most, our own cries of ``Why?" cannot match His. But we can utter the same submissive word ``nevertheless ..." (Matt. 26:39).

Progression toward submission confers another blessing: an enhanced capacity for joy. Counseled President Brigham Young, ``If you want to enjoy exquisitely, become a Latter-day Saint, and then live the doctrine of Jesus Christ" (in Journal of Discourses, 18:247).

John Taylor indicated that the Lord may even choose to wrench our very heartstrings (see Journal of Discourses, 14:360). If our hearts are set too much upon the things of this world, they may need to be wrenched, or broken, or undergo a mighty change (see Alma 5:12).

Long before that, however, as Jesus declared, we must ``settle this in [our] hearts" that we will do what He asks of us (JST, Luke 14:28). President Young further counseled us ``to submit to the hand of the Lord, ...\footnote{to his providences} and acknowledge his hand in all things, ...\footnote{and always be willing that he should dictate, though it should take your houses, your property, your wives and children, your parents, your lives, or anything else you have upon the earth} then you will be exactly right; and until you come to that point, you cannot be entirely right. That is what we have to come to" (in Journal of Discourses, 5:352).

\subsection{From the Beginning, October 1993}

President Joseph F. Smith observed that amid this diffusion certain laws and rites were ``carried by the posterity of Adam into all lands, and continued with them, more or less pure, to the flood, and through Noah, ...\footnote{
who was a ``preacher of righteousness,"  
} to those who succeeded him, spreading out into all nations and countries. ...\footnote{Adam and Noah being the first of their dispensations to receive them from God.} What wonder, then, that we should find relics of Christianity, so to speak, among ...\footnote{the heathens and} nations who know not Christ, and whose histories date back ...\footnote{beyond the days of Moses, and even} beyond the flood, independent of and apart from the records of the Bible" (in Journal of Discourses, 15:325; see also Alma 29:8).

Some local leaders rebelled, as when one, who loved his preeminence, refused to receive the brethren (see 3 Jn. 1:9–10).

No wonder President Brigham Young observed: ``It is said the Priesthood was taken from the Church, but it is not so, the Church went from the Priesthood" (in Journal of Discourses, 12:69).

Self-siftings do occur. President George Q. Cannon observed in 1875:

``I am thankful that God allows those who do not keep his commandments to fall away, so that his Church may be cleansed, and, in this respect, this Church is different from any other that is upon the earth. ...\footnote{A man may practice iniquity and do wrong in other churches, and he may cover it up for years, and nobody, or probably but a few—himself, his God, and a few others—be aware of this wrong, and he may pass along and nobody ever imagine that there is anything wrong with him. But it is not so in the Church of Jesus Christ of Latter-day Saints—no man can stand in this Church, or retain the Spirit of God and continue in a course of hypocrisy for any length of time. God will tear away the covering of lies and expose the wrong; he will leave the transgressor to himself, and the strength that he formerly had, which enabled him to stand and maintain his associations with the people of God, will be taken away from him, and he will be left to go down to destruction unless he repents. It is true that the Lord has said that the tares shall grow with the wheat until harvest, but it is not said that tares will not be plucked up from time to time, for if it were not so they would overpower and choke out the wheat.} The sifting or weeding process has been going on from the commencement of this Church until the present time\footnote{; hence it is that the leaders of this Church are stirred up in their feelings from time to time to call upon the people to repent. They understand clearly that unless there is a godly life and conversation corresponding with our profession, this people would soon fall into darkness and error, and stray from the path of righteousness.}" (in Journal of Discourses, 18:84).

\subsection{Called and Prepared from the Foundation of the World, April 1986}

Before the Restoration, the void was very real. Prior to meeting Joseph Smith, Brigham Young said he would have crawled around the earth on his hands and knees to meet someone like Moses who could tell him anything ``about God and heaven."\footnote{Before I had made a profession of religion, I was thought to be an infidel by the Christians, because I could not believe their nonsense. The secret feeling of my heart was that I would be willing to crawl around the earth on my hands and knees, to see such a man as was Peter, Jeremiah, Moses, or any man that could tell me anything about God and heaven.} (In Journal of Discourses, 8:228.) Through Joseph Smith we have additional pages from Moses about God and heaven. We have only to reach to the bookshelf or go to priesthood meeting. Perhaps the way is almost too easy and too simple; we might be more appreciative if on hands and knees. (See 1 Ne. 17:41.) Only by searching the scriptures, not using them occasionally as quote books, can we begin to understand the implications as well as the declarations of the gospel.

\subsection{Settle This in Your Hearts, October 1992}

When the determination is first made to begin to be more spiritually settled, there is an initial vulnerability: it is hard to break with the past. But once we begin, we see how friends who would hold us back spiritually are not true friends at all. Any chiding from them reflects either resentment or unconscious worry that somehow they are being deserted. In any attempt to explain to them, our tongue is able to speak only ``the smallest part." (Alma 26:16.) We continue to care for them, but we care for our duty to God more. Brigham Young counseled candidly: ``Some do not understand duties which do not coincide with their natural feelings and affections. ...\footnote{Do you comprehend that statement? I have tried to tell you; but I am sometimes at a loss to convey a correct understanding with words. I should have the language of angels to enable me to exactly convey my ideas, and that would require an audience who understand that language.} There are duties which are above affection." (Journal of Discourses, 7:65.)