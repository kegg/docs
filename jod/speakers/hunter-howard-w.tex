\section{Howard W. Hunter}

\subsection{The Golden Thread of Choice, October 1989}

Brigham Young once said: ``The volition of [man]\footnote{the creature} is free; this is a law of their existence, and the Lord cannot violate his own law; were he to do that, he would cease to be God. …\footnote{He has placed life and death before his children, and it is for them to choose. If they choose life, they receive the blessings of life; if they chose death, they must abide the penalty.} This is a law which has always existed from all eternity, and will continue to exist throughout all the eternities to come. Every intelligent being must have the power of choice." (In Journal of Discourses, 11:272.)

\subsection{Fast Day, October 1985}

We do not know when fasting was adopted in the Church as a regular observance, but there are records that indicate that some fast meetings were held in the Kirtland Temple on the first Thursday of each month in the year 1836. There is no indication that these fasts were associated with donations to the poor, except a remark made by Brigham Young more than thirty years later in the Old Tabernacle in Salt Lake City. He had this to say:

``You know that the first Thursday in each month we hold as a fast day. How many here know the origin of this day? Before tithing was paid, the poor were supported by donations. They came to Joseph and wanted help, in Kirtland, and he said there should be a fast day, which was decided upon. It was to be held once a month, as it is now, and all that would have been eaten that day, of flour, or meat, or butter, or fruit, or anything else, was to be carried to the fast meeting and put into the hands of a person selected for the purpose of taking care of it and distributing it among the poor" (Journal of Discourses, 12:115.)

\subsection{The Gospel—A Global Faith, October 1991}

Brigham Young once said about such a broad and stimulating concept of religion: ``For me, the plan of salvation must ...\footnote{must be a system that is pure and holy in all its points; it must reveal things that no other Church or kingdom can reveal; it must} circumscribe [all] the knowledge that is upon the face of the earth, or it is not from God. Such a plan incorporates every system of true doctrine on the earth, whether it be ecclesiastical, moral, philosophical, or civil: it incorporates all good laws that have been made from the days of Adam until now; it swallows up the laws of nations, for it exceeds them all in knowledge and purity; it circumscribes the doctrines of the day, and takes from the right and the left, and brings all truth together in one system, and leaves the chaff to be scattered hither and thither." (Journal of Discourses, 7:148.)

\subsection{The Tabernacle, October 1975}

Elder George Q. Cannon stood at this pulpit after the building was completed but not yet dedicated and talked about missionary work. His words seem to echo from the past what our president is saying to us today. He said: ``Our Elders have gone to the Eastern States by hundreds to lift up their warning voices to the people concerning the things which God is doing and is about to do in the midst of the inhabitants of the earth. For this purpose they go to Europe, to the West, to the Islands of the Pacific, to Asia and Africa, and they will yet traverse every country on the face of the whole earth. The millions of Asia will yet hear the glad tidings of salvation from the Elders of Israel ...\footnote{The yoke of bondage is being broken and the nations are being freed from the grasp of despotism and tyranny. Japan now opens her ports; China begins to extend her invitation to western civilization,} and the time is near at hand when the sound of this Gospel, proclaimed by the Elders of Israel, will re-echo from one end of the earth to the other, for it must be preached as a witness unto all nations." (Journal of Discourses 13:53.)

\subsection{God Will Have a Tried People, April 1980}

One hundred fifty years of Church history provide us with a lesson that when resistance and opposition are greatest, our faith, commitment, and growth have the greatest opportunity for advancement; when opposition is least, the tendency is to be complacent and lose faith. President Brigham Young said: ``Let any people enjoy peace and quietness, unmolested, undisturbed,—never be persecuted for their religion, and they are very likely to neglect their duty, to become cold and indifferent, and lose their faith" (in Journal of Discourses, 7:42). This lesson, which applies to the Church collectively, also applies to individuals.