\section{Dallin H. Oaks}

\subsection{Joseph, the Man and the Prophet, April 1996}

Like most other leaders on the frontier, Joseph Smith did not shrink from physical confrontation, and he had the courage of a lion. Once he was kidnapped by two men who held cocked pistols to his head and repeatedly threatened to shoot him if he moved a muscle. The Prophet endured these threats for a time and then snapped back, ``Shoot away; I have endured so much persecution and oppression that I am sick of life; why then don't you shoot, and have done with it, instead of talking so much about it?" (in Journal of Discourses, 2:167; see also History of the Church, 5:440).

Men who knew Joseph best and stood closest to him in Church leadership loved and sustained him as a prophet. His brother Hyrum chose to die at his side. John Taylor, also with him when he was murdered, said: ``I testify before God, angels, and men, that he was a good, honorable, virtuous man ... —that his private and public character was unimpeachable—and that he lived and died as a man of God" (The Gospel Kingdom, [1987], 355; see also D\&C 135:3). Brigham Young declared: ``I do not think that a man lives on the earth that knew [Joseph Smith] any better than I did; and I am bold to say that, Jesus Christ excepted, no better man ever lived or does live upon this earth" (in Journal of Discourses, 9:332).

\subsection{Apostasy and Restoration, April 1995}

In the theology of the restored church of Jesus Christ, the purpose of mortal life is to prepare us to realize our destiny as sons and daughters of God—to become like Him. Joseph Smith and Brigham Young both taught that ``no man ... can know himself unless he knows God, and he can not know God unless he knows himself" (in Journal of Discourses, 16:75; see also The Words of Joseph Smith, ed. Andrew F. Ehat and Lyndon W. Cook, Provo: Religious Studies Center, Brigham Young University, 1980, p. 340). The Bible describes mortals as ``the children of God" and as ``heirs of God, and joint-heirs with Christ" (Rom. 8:16–17). It also declares that ``we suffer with him, that we may be also glorified together" (Rom. 8:17) and that ``when he shall appear, we shall be like him" (1 Jn. 3:2). We take these Bible teachings literally. We believe that the purpose of mortal life is to acquire a physical body and, through the atonement of Jesus Christ and by obedience to the laws and ordinances of the gospel, to qualify for the glorified, resurrected celestial state that is called exaltation or eternal life.