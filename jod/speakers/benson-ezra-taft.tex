\section{Ezra Taft Benson}

\subsection{Prepare for the Days of Tribulation, October 1980}

President Brigham Young said, ``If you are without bread, how much wisdom can you boast, and of what real utility are your talents, if you cannot procure for yourselves and save against a day of scarcity those substances designed to sustain your natural lives?" (In Journal of Discourses, 8:68.)

From the standpoint of food production, storage, handling, and the Lord’s counsel, wheat should have high priority. ``There is more salvation and security in wheat," said Orson Hyde years ago, ``than in all the political schemes of the world" (in Journal of Discourses, 2:207). Water, of course, is essential. Other basics could include honey or sugar, legumes, milk products or substitutes, and salt or its equivalent. The revelation to produce and store food may be as essential to our temporal welfare today as boarding the ark was to the people in the days of Noah.

\subsection{Worthy of All Acceptation, October 1978}

Once the families in the Church become organized as the prophet has counseled, and after we have done all we can as a church and as family organizations to identify our progenitors, then perhaps we may qualify for this prophetic blessing spoken of by President Brigham Young:

``You will enter into the Temple of the Lord and begin to offer up ordinances before the Lord for your dead. ...\footnote{Says this or that man, I want to save such a person—I want to save my father; and he straightway goes forth in the ordinance of baptism, and is confirmed, and washed, and anointed, and ordained to the blessings of the holy Priesthood for his ancestors.} Before this work is finished, a great many of the Elders of Israel in Mount Zion will become pillars in the Temple of God, to go no more out: they will eat and drink and sleep there; and they will often have occasion to say—`Somebody came into the Temple last night; we did not know who he was, but he was no doubt a brother, and told us a great many things we did not before understand. \textit{He gave us the names of a great many of our forefathers that are not on record}, and he gave me my true lineage and the names of my forefathers for hundreds of years back. He said to me, `You and I are connected in one family; there are the names of your ancestors; take them and write them down, and be baptised and confirmed, and save such and such ones, and receive of the blessings of the eternal Priesthood for such and such an individual, as you do for yourselves.' This is what we are going to do for the inhabitants of the earth. When I look at it, I do not want to rest a great deal, but be industrious all the day long; for when we come to think upon it, we have no time to lose, for it is a pretty laborious work." (Journal of Discourses, 6:295; italics added.)

``Let us go to and attend to our ordinances, then when we go to the spirit world and meet with father, mother, brother or sister they cannot rise up and accuse us of negligence. ...\footnote{I have attended to the ordinances for a great many of my friends, and I want you to do the same, so that when we get to the other side of the veil we may look back and be satisfied. This power has been placed in the hands of the Latter-day Saints, then let us go forth and use it for the salvation of the living and the dead. With regard to the unbelief of the world, it will not make the truth of God without effect.} These [temple] ordinances have been revealed to us; we understand them, and unless we attend to them we shall fall under condemnation." (Wilford Woodruff, in Journal of Discourses, 13:327.)

\subsection{Our Priceless Heritage, Ocotber 1976}

We know the signers of the sacred Declaration of Independence and the Founding Fathers, with George Washington at their head, have made appearance in holy places. Apostle Wilford Woodruff was president of the St. George Temple at the time of their appearance and testified that the founders of our republic declared this to him: ``We laid the foundation of the government you now enjoy, and we never apostatized from it, but we remained true to it and were faithful to God." (Journal of Discourses, 19:229.)

\subsection{Cleansing the Inner Vessel, April 1986}

The plaguing sin of this generation is sexual immorality. This, the Prophet Joseph said, would be the source of more temptations, more buffetings, and more difficulties for the elders of Israel than any other. (See Journal of Discourses, 8:55.)\footnote{Some of our Missionaries, after an absence of two or three years, return with their eyes cast down: their countenances are fallen. I wish you to take such a course that you can come home with your heads up. Keep yourselves clean, from the crowns of your heads to the soles of your feet; be pure in heart—otherwise you will return bowed down in spirit and with a fallen countenance, and will feel as though you never could rise again. When the Quorum of the Twelve was first organized, Joseph said that the Elders of Israel, and particularly the Twelve Apostles, would receive more temptations, be more buffeted, and have greater difficulty to escape the evil thrown in their way by females than by any other means. This is one of Satan's most powerful auxiliaries with which to weaken the influence of the ministers of Christ, and bring them down from their high position and calling into darkness, shame, and disgrace. You will have to guard more strictly against that than against any other evil that may beset you. Make up your minds not to yield, for one moment, to the subtle insinuations of the animal propensities of your natures while you are absent on the Lord's errands. Rather, suffer your heads to be taken from your shoulders than to sacrifice your honor, violate your covenants, and forfeit the sacred trust reposed in you.}

\subsection{Valiant in the Testimony of Jesus, April 1982}

President Brigham Young revealed that on one occasion he was tempted to be critical of the Prophet Joseph Smith regarding a certain financial matter. He said that the feeling did not last for more than perhaps thirty seconds. That feeling, he said, caused him great sorrow in his heart. The lesson he gave to members of the Church in his day may well be increased in significance today because the devil continues more active:

``I clearly saw and understood, by the spirit of revelation manifested to me, that if I was to harbor a thought in my heart that Joseph could be wrong in anything, I would begin to lose confidence in him, and that feeling would grow from step to step, and from one degree to another, until at last I would have the same lack of confidence in his being the mouthpiece for the Almighty. ...\footnote{and I would be left, as brother Hooper observed, upon the brink of the precipice, ready to plunge into what we may call the gulf of infidelity, ready to believe neither in God nor His servants, and to say that there is no God, or, if there is, we do not know anything about Him; that we are here, and by and by shall go from here, and that is all we shall know. Such persons are like those whom the Apostle calls ``As natural brute beasts, made to be taken and destroyed." Though I admitted in my feelings and knew all the time that Joseph was a human being and subject to err, still it was none of my business to look after his faults.}

``I repented of my unbelief, and that too, very suddenly; I repented about as quickly as I committed the error. It was not for me to question whether Joseph was dictated by the Lord at all times and under all circumstances. ...\footnote{or not. I never had the feeling for one moment, to believe that any man or set of men or beings upon the face of the whole earth had anything to do with him, for he was superior to them all, and held the keys of salvation over them. Had I not thoroughly understood this and believed it, I much doubt whether I should ever have embraced what is called ``Mormonism." He was called of God; God dictated him, and if He had a mind to leave him to himself and let him commit an error, that was no business of mine. And it was not for me to question it, if the Lord was disposed to let Joseph lead the people astray, for He had called him and instructed him to gather Israel and restore the Priesthood and kingdom to them. }

``It was not my prerogative to call him in question with regard to any act of his life. He was God's servant, and not mine. He did not belong to the people but to the Lord, and was doing the work of the Lord." (In Journal of Discourses, 4:297.)

\subsection{A Principle with a Promise, April 1983}

In the absence of a temple, the first School of the Prophets was held in a small room in the home of Bishop Newel K. Whitney. Brigham Young was one of the early participants in this school, and he described a scene which frequently presented itself during meetings:

``The brethren came to that place for hundreds of miles to attend school in a little room probably no larger than eleven by fourteen. When they assembled together in this room after breakfast, the first they did was to light their pipes, and, while smoking, talk about the great things of the kingdom, ...\footnote{and spit all over the room} and as soon as the pipe was out of their mouths a large chew of tobacco would then be taken. Often when the Prophet entered the room to give the school instructions he would find himself in a cloud of tobacco smoke. This, and the complaints of his wife at having to clean [the]\footnote{so filthy a} floor, made the Prophet think upon the matter, and he inquired of the Lord relating to the conduct of the Elders in using tobacco." (Journal of Discourses, 12:158.)

\subsection{A Marvelous Work and a Wonder, April 1980}

I call on all inactive priesthood holders—you who, for reasons best known to yourselves, are disassociated from your quorums and church. You have formed new affiliations, and now some of you have become disinterested in the Church and no longer conform to its standards. Unhappily, many of your families tread in your paths and follow your examples. Brethren, when we fail to be true to our priesthood promises, the price we and our loved ones are forced to pay might well be entitled ``the high cost for low living." What a blessing you would be to your wives and children if you would harmonize your lives with your covenants. O, brethren of the priesthood, how we need your support, affiliation, and strength! Do not desert the cause of God at a time when the conflict is most imminent. Make President John Taylor’s slogan your commitment: ``The kingdom of God or nothing!" (in Journal of Discourses, 6:26).

\subsection{Prepare Ye, October 1973}

For over 100 years we have been admonished to store up grain. ``Remember the counsel that is given," said Elder Orson Hyde, ```...\footnote{Now, we have the gift of God, and that is the gift of wise counsel—of good counsel given unto us for the purpose of self-preservation. Will we, by any reason, by any craft, by any device, by any machinations, by any swerving from our purpose, lose that gift? Remember that if we are upon the enemies' ground, the gift that is given to us may be destroyed or taken from us forever; and probably the time may be that you and I may not have the counsel of the servants of God from day to day. If it is necessary, however, we may have it; and if it is not; remember it, ye Latter-day Saints, and everybody that fears God and serves Him with full purpose of heart! Remember the counsel that is given,} Store up all your grain,' and take care of it! … And I tell you it is almost as necessary to have bread to sustain the body as it is to have food for the spirit; for the one is as necessary as the other to enable us to carry on the work of God upon the earth." (Journal of Discourses, vol. 5, p. 17.) And he also said: ``There is more salvation and security in wheat, than in all the political schemes of the world. ...\footnote{and also more power in it than in all the contending armies of the nations}" (JD, vol. 2, p. 207.)

\subsection{Civic Standards for the Faithful Saints, April 1972}

President Wilford Woodruff said:

``Now I have thought many times that some of those ancient kings that were raised up, had in some respects more regard for the carrying out of some of these principles and laws, than even the Latter-day Saints have in our day. I will take as an ensample Cyrus. ...\footnote{on account of his temperance. He was one of the kings of the Medes and Persians. I believe his father was a Persian and his mother a Mede} To trace the life of Cyrus from his birth to his death, whether he knew it or not, it looked as though he lived by inspiration in all his movements. He began with that temperance and virtue which would sustain any Christian country or any Christian king. \footnote{And even when he was sent in his youth to his grandfather Astyages, the king of the Medes, he showed that he had been carefully brought up, and he followed his early training in a great measure throughout his life; while as king or leader of the Median armies, he conquered nearly the whole world—in fact I do not know that he ever lost a battle. His grandfather was living in luxury, and when young Cyrus was sent to him he offered to serve him as a butler—only he didn't do as butler's sometimes do—that is, taste the wine before putting it on the table. Cyrus, when offered wine, said, ``I am afraid it is poison." ``You are afraid it is poison?" ``What makes you think it poison?" ``Why, because I have seen it make you and some of the princes act very strange, you would stagger and act very curious." He followed this principle of temperance during his whole life. Before a battle he offered sacrifices to the Gods; when he finished a battle and had a victory he did the same thing. I have been struck in reading his history with the course he took in this matter. He would never enter into revelry or debauchery over the nations he had conquered. He taught such principles until the day of his death. Before he died he told those by whom he was surrounded, that he did not want his body put into a gold coffin or a silver coffin; he simply desired his body to be laid in the dust and covered with the earth.} Many of these principles followed him, and I have thought many of them were worthy, in many respects, the attention of men who have the Gospel of Jesus Christ." (Journal of Discourses, vol. 22, p. 207.)

\subsection{Life Is Eternal, April 1971}

The spirit world is not far away. Sometimes the veil between this life and the life beyond becomes very thin. Our loved ones who have passed on are not far from us. One great spiritual leader asked, ``But where is the spirit world?" and then answered his own question. ``It is here." ``Do [spirits] go beyond the boundaries of this organized earth? No, they do not. They are brought forth upon this earth, for the express purpose of inhabiting it to all eternity." "... when the spirits leave their bodies they are in the presence of our Father and God; they are prepared then to see, hear and understand spiritual things. ...\footnote{But where is the spirit world? It is incorporated within this celestial system. Can you see it with your natural eyes? No. Can you see spirits in this room? No. Suppose the Lord should touch your eyes that you might see, could you then see the spirits? Yes, as plainly as you now see bodies, as did the servant of Elijah.} If the Lord would permit it, and it was His will that it should be done, you could see the spirits that have departed from this world, as plainly as you now see bodies with your natural eyes. ...\footnote{as plainly as brothers Kimball and Hyde saw those wicked disembodied spirits in Preston, England. They saw devils there, as we see one another; they could hear them speak, and knew what they said. Could they hear them with the natural ear? No. Did they see those wicked spirits with their natural eyes? No. They could not see them the next morning, when they were not in the spirit; neither could they see them the day before, nor at any other time; their spiritual eyes were touched by the power of the Almighty.}" (Brigham Young, in Journal of Discourses, vol. 3, pp. 367–69.)