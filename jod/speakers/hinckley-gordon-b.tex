\section{Gordon B. Hinckley}

\subsection{Small Acts Lead to Great Consequences, April 1984}

There stood once on the grounds right here, before ever this building was constructed, a bowery—a rather crude structure in which the Saints met in those days of their poverty. In September of 1857, there was presented in that old bowery on a Sunday afternoon, what was really the concluding act of a drama of great tragedy.

On that Sunday Brigham Young was conducting a meeting and introduced to the congregation a man who appeared to be old and infirm and weary of life.

Said President Brigham Young to the congregation:

``Brother Thomas B. Marsh, formerly the President of the Quorum of the Twelve Apostles, has now come to us, after an absence of nearly nineteen years. He is on the stand to-day, and wishes to make a few remarks to the congregation. ...\footnote{You will comprehend the purport of the remarks he wishes to make, by my relating a part of his conversation with me yesterday.}

``He came into my office and wished to know whether I could be reconciled to him, and whether there could be a reconciliation between himself and the Church of the living God. He reflected for a moment and said, I am reconciled to the Church, but I want to know whether the Church can be reconciled to me.

``He is here," said President Young, ``and I want him to say what he may wish to. ... Brethren and sisters, I now introduce to you Brother Thomas B. Marsh. When the Quorum of the Twelve was first organized, he was appointed to be their President."

Brother Marsh rose to the pulpit. This man, who was named the first President of the Council of the Twelve Apostles and to whom the Lord had spoken in so marvelous a manner, as recorded in section 112 of the Doctrine and Covenants—which I wish you would read—said to the people:

``I do not know that I can make all this vast congregation hear and understand me. My voice never was very strong, but it has been very much weakened of late years by the afflicting rod of Jehovah. He loved me too much to let me go without whipping. I have seen the hand of the Lord in the chastisement which I have received. I have seen and known that it has proved he loved me; for if he had not cared anything about me, he would not have taken me by the arm and given me such a shaking.

``If there are any among this people who should ever apostatize and do as I have done, prepare your backs for a good whipping, if you are such as the Lord loves. But if you will take my advice, you will stand by the authorities; but if you go away and the Lord loves you as much as he did me, he will whip you back again.

``Many have said to me," he continued, ```How is it that a man like you, who understood so much of the revelations of God as recorded in the Book of Doctrine and Covenants, should fall away?’ I told them not to feel too secure, but to take heed lest they also should fall; for I had no scruples in my mind as to the possibility of men falling away."

He continued, ``I can say, in reference to the Quorum of the Twelve, to which I belonged, that I did not consider myself a whit behind any of them, and I suppose that others had the same opinion; but, let no one feel too secure; for, before you think of it, your steps will slide. You will not then think nor feel for a moment as you did before you lost the Spirit of Christ; for when men apostatize, they are left to grovel in the dark." (Journal of Discourses, 5:206.)

Speaking in a voice that was difficult to hear, and appearing as an old man when he was actually only fifty-seven years of age, he spoke of the travails through which he had passed before he had finally made his way to the valley of the Great Salt Lake and asked that he might be baptized again into the Church.

I wondered, as I read that story so filled with pathos, what had brought him to this sorry state. I discovered it, in the Journal of Discourses, in a talk given to the Saints in this same bowery the year before by George A. Smith. I think, if you’ll bear with me for a minute or two, it is worth the telling to illustrate to all of us the need to be careful in dealing with small matters which can lead to great consequences.

According to the account given by George A. Smith, while the Saints were in Far West, Missouri, ``the wife of Thomas B. Marsh, who was then President of the Twelve Apostles, and Sister Harris concluded they would exchange milk, in order to make a little larger cheese than they otherwise could. To be sure to have justice done, it was agreed that they should not save the strippings (to themselves), but that the milk and strippings should all go together.

Now for you who have never been around a cow, I should say that the strippings came at the end of the milking and were richer in cream.

``Mrs. Harris, it appeared, was faithful to the agreement and carried to Mrs. Marsh the milk and strippings, but Mrs. Marsh, wishing to make some extra good cheese, saved a pint of strippings from each cow and sent Mrs. Harris the milk without the strippings."

A quarrel arose, and the matter was referred to the home teachers. They found Mrs. Marsh guilty of failure to keep her agreement. She and her husband were upset and, ``an appeal was taken from the teacher to the bishop, and a regular Church trial was had.” President Marsh did not consider that the bishop had done him and his lady justice for they (that is, the bishop’s court) decided that the strippings were wrongfully saved, and that the woman had violated her covenant.

``Marsh immediately took an appeal to the High Council, who investigated the question with much patience, and," says George A. Smith, “I assure you they were a grave body. Marsh being extremely anxious to maintain the character of his wife, ...\footnote{as he was the President of the Twelve Apostles, and a great man in Israel,} made a desperate defence, but the High Council finally confirmed the bishop's decision.

``Marsh, not being satisfied, took an appeal to the First Presidency of the Church, and Joseph and his counselors had to sit upon the case, and they approved the decision of the High Council.

``This little affair," Brother Smith continues, ``kicked up a considerable breeze, and Thomas B. Marsh then declared that he would sustain the character of his wife even if he had to go to hell for it.

``The then President of the Twelve Apostles, the man who should have been the first to do justice and cause reparation to be made for wrong, committed by any member of the family, took that position, and what next? He went before a magistrate and swore that the ‘Mormons’ were hostile towards the state of Missouri.

``That affidavit brought from the government of Missouri an exterminating order, which drove some 15,000 Saints from their homes and habitations, and some thousands perished through suffering the exposure consequent on this state of affairs." (Journal of Discourses, 3:283–84.) Such is George A. Smith’s account.

\subsection{Live the Gospel, October 1984}

One hundred years ago at the October 1884 conference, in this same Tabernacle, George Q. Cannon, Counselor in the First Presidency, standing where I now stand, said to those here assembled:

``If I could speak so that the whole world would hear the utterance I would like to sound it in the ears of all mortal men—that there is no power that will ever be permitted to array itself, or to combine itself against this work of our God, to retard its onward progress from this time forward until the full consummation will be achieved—that is, if the Latter-day Saints themselves are faithful to God, if they will keep the commandments of God, if they will sanctify themselves and cleanse themselves from sin, and live pure and holy lives. If they will do this, then the success and the triumph and the continued growth and advancement of this kingdom ...\footnote{and the continued maintenance of these valleys and these mountains} are assured unto us as a people. There is no doubt of it. I say in the name of Jesus Christ, that it will be so." (Journal of Discourses, 25:325.)

\subsection{Live Up to Your Inheritance, October 1983}

It is a time for education. The world that lies ahead of you will be fiercely competitive. Now is the time to train yourselves for possible future responsibilities.

Education is a tradition that has come down from our early history. We believe in the training of our youth, girls as well as boys. Brigham Young once said, ``We have sisters here who, if they had the privilege of studying, would make just as good mathematicians or accountants as any man." (Journal of Discourses, 13:61.)

\subsection {God Is at the Helm, April 1994}

Concerning the sustaining of officers, President John Taylor once said:

``We hold up our right hand when voting in token before God that we will sustain those for whom we vote; and if we cannot feel to sustain them we ought not to hold up our hands, because to do this, would be to act the part of hypocrites. ...\footnote{And the question naturally arises, how far shall we sustain them? Or in other words, how far are we at liberty to depart from this covenant which we make before each other and before our God?} For when we lift up our hands in this way, it is in token to God that we are sincere in what we do, and that we will sustain the parties we vote for. \footnote{This is the way I look at these things. How far then should we sustain them, and how far should we not? This is a matter of serious importance to us;} If we agree to do a thing and do not do it, we become covenant breakers and violators of our obligations, which are, perhaps, as solemn and binding as anything we can enter into" (in Journal of Discourses, 21:207).