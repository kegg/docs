\section{Victor L. Brown}

\subsection{Prepare Every Needful Thing, October 1980}

``View the actions of the Latter-day Saints on this matter, and their neglect of the counsel given; and suppose the Lord would allow these insects to destroy our crops this season and the next, what would be the result? I can see death, misery and want on the faces of this people. But some may say, `I have faith the Lord will turn them away.' What ground have we to hope this? Have I any good reason to say to my Father in heaven, `Fight my battles,' when He has given me the sword to wield, the arm and the brain that I can fight for myself? Can I ask Him to fight my battles and sit quietly down waiting for Him to do so? I cannot. I can pray the people to hearken to wisdom, to listen to counsel; but to ask God to do for me that which I can do for myself is preposterous to my mind. Look at the Latter-day Saints. We have had our fields laden with grain for years; and if we had been so disposed, our bins might have been filled to overflowing, and with seven years' provisions on hand we might have disregarded the ravages of these insects, and have gone to the canyon and got our lumber, procured the materials, and built up and beautified our places, instead of devoting our time to fighting and endeavoring to replace that which has been lost through their destructiveness. We might have made our fences, improved our buildings, beautified Zion, let our ground rest, and prepared for the time when these insects would have gone. But now the people are running distracted here and there. ...\footnote{I do not wish to condemn them. I wish all the justification that can be brought to them. But I look at them as they are.} They are in want and in trouble, and they are perplexed. They do not know what to do. They have been told what to do, but they did not hearken to this counsel." (In Journal of Discourses, 12:240–41.)

President Young goes on to say: ``We must learn to listen to the whispering of the Holy Spirit, and the counsels of the servants of God, until we come to the unity of the faith. If we had obeyed counsel we would have had granaries today, and they would have been full of grain; and we would have had wheat and oats and barley for ourselves and for our animals, to last us for years." (In Journal of Discourses, 12:241.)

Quoting further from President Young: ``When Moses was on the mount they [the Israelites] went to Aaron and inquired where Moses was, and demanded gods to go before them. And Aaron told them to bring him their ear rings and their jewelry, and they did so, and he made of them a golden calf; and the people ran around it, and said these be the gods which brought us out of the land of Egypt. How much credit was due to them? Just as much as to us, for not saving our grain when we had an abundance, and, when grasshoppers come, crying, `Lord turn them away and save us.' It is just as consistent as for a man on board a steamboat on the wide ocean to say, I will show you what faith I have, and then to jump overboard, crying, `Lord save me.' It may not seem so daring; but is it any more inconsistent than to throw away and waste the substance the Lord has given us, and when we come to want, crying to Him for what we have wasted and squandered? The Lord has been blessing us all the time, and He asks us why we have not been blessing ourselves." (In Journal of Discourses, 12:243.)

``We have seen one grasshopper war before this. Then we had two years of it. We are having two years now. Suppose we have good crops next year, the people will think less of this visitation than they do now; and still less the next year; until in four or five years it will be almost gone from their minds. We are capable of being perfectly independent of these insects. If we had thousands on thousands of bushels of wheat, rye, and barley, and corn we might have said to them, [that is, the insects] `you may go, we are not going to plant for you.' Then we could have plowed up the ground, put in the manure, and let the land rest, and the grasshoppers would not have destroyed the fruits of our labors which could have been directed to the beautifying of Zion and making our habitations places of loveliness." (In Journal of Discourses, 12:242.)